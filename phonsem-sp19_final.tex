\documentclass[,doc,floatsintext]{apa6}
\usepackage{lmodern}
\usepackage{amssymb,amsmath}
\usepackage{ifxetex,ifluatex}
\usepackage{fixltx2e} % provides \textsubscript
\ifnum 0\ifxetex 1\fi\ifluatex 1\fi=0 % if pdftex
  \usepackage[T1]{fontenc}
  \usepackage[utf8]{inputenc}
\else % if luatex or xelatex
  \ifxetex
    \usepackage{mathspec}
  \else
    \usepackage{fontspec}
  \fi
  \defaultfontfeatures{Ligatures=TeX,Scale=MatchLowercase}
\fi
% use upquote if available, for straight quotes in verbatim environments
\IfFileExists{upquote.sty}{\usepackage{upquote}}{}
% use microtype if available
\IfFileExists{microtype.sty}{%
\usepackage{microtype}
\UseMicrotypeSet[protrusion]{basicmath} % disable protrusion for tt fonts
}{}
\usepackage{hyperref}
\hypersetup{unicode=true,
            pdftitle={Vowel harmony with QFLFP},
            pdfauthor={Eileen Blum},
            pdfkeywords={keywords},
            pdfborder={0 0 0},
            breaklinks=true}
\urlstyle{same}  % don't use monospace font for urls
\usepackage{graphicx,grffile}
\makeatletter
\def\maxwidth{\ifdim\Gin@nat@width>\linewidth\linewidth\else\Gin@nat@width\fi}
\def\maxheight{\ifdim\Gin@nat@height>\textheight\textheight\else\Gin@nat@height\fi}
\makeatother
% Scale images if necessary, so that they will not overflow the page
% margins by default, and it is still possible to overwrite the defaults
% using explicit options in \includegraphics[width, height, ...]{}
\setkeys{Gin}{width=\maxwidth,height=\maxheight,keepaspectratio}
\setlength{\emergencystretch}{3em}  % prevent overfull lines
\providecommand{\tightlist}{%
  \setlength{\itemsep}{0pt}\setlength{\parskip}{0pt}}
\setcounter{secnumdepth}{5}
% Redefines (sub)paragraphs to behave more like sections
\ifx\paragraph\undefined\else
\let\oldparagraph\paragraph
\renewcommand{\paragraph}[1]{\oldparagraph{#1}\mbox{}}
\fi
\ifx\subparagraph\undefined\else
\let\oldsubparagraph\subparagraph
\renewcommand{\subparagraph}[1]{\oldsubparagraph{#1}\mbox{}}
\fi

%%% Use protect on footnotes to avoid problems with footnotes in titles
\let\rmarkdownfootnote\footnote%
\def\footnote{\protect\rmarkdownfootnote}

%%% Change title format to be more compact
\usepackage{titling}

% Create subtitle command for use in maketitle
\newcommand{\subtitle}[1]{
  \posttitle{
    \begin{center}\large#1\end{center}
    }
}

\setlength{\droptitle}{-2em}
  \title{Vowel harmony with QFLFP}
  \pretitle{\vspace{\droptitle}\centering\huge}
  \posttitle{\par}
  \author{Eileen Blum\textsuperscript{1}}
  \preauthor{\centering\large\emph}
  \postauthor{\par}
  \date{}
  \predate{}\postdate{}

\shorttitle{VHQFLFP}
\authornote{Eileen Blum is a graduate student in the Department of Linguistics at Rutgers University.


Correspondence concerning this article should be addressed to Eileen Blum, 18 Seminary Place, New Brunswick, NJ 08901. E-mail: eileen.blum@rutgers.edu}
\affiliation{
\vspace{0.5cm}
\textsuperscript{1} Rutgers University}
\keywords{keywords\newline\indent Word count: X}
\usepackage{csquotes}
\usepackage{upgreek}
\captionsetup{font=singlespacing,justification=justified}

\usepackage{longtable}
\usepackage{lscape}
\usepackage{multirow}
\usepackage{tabularx}
\usepackage[flushleft]{threeparttable}
\usepackage{threeparttablex}

\newenvironment{lltable}{\begin{landscape}\begin{center}\begin{ThreePartTable}}{\end{ThreePartTable}\end{center}\end{landscape}}

\makeatletter
\newcommand\LastLTentrywidth{1em}
\newlength\longtablewidth
\setlength{\longtablewidth}{1in}
\newcommand{\getlongtablewidth}{\begingroup \ifcsname LT@\roman{LT@tables}\endcsname \global\longtablewidth=0pt \renewcommand{\LT@entry}[2]{\global\advance\longtablewidth by ##2\relax\gdef\LastLTentrywidth{##2}}\@nameuse{LT@\roman{LT@tables}} \fi \endgroup}


\usepackage{tipa}
\usepackage{gb4e}
\noautomath
\usepackage{tikz}
\usetikzlibrary{matrix}
\tikzset{marked/.style={draw=none, fill=none}}
\usetikzlibrary{topaths}
\usepackage{mathptmx}
\usepackage{moresize}
\usepackage{multicol}
\setlength{\parindent}{2em}
\def\defeq{\mathrel{\buildrel \mbox{\footnotesize def} \over =}}

\usepackage{amsthm}
\newtheorem{theorem}{Theorem}[section]
\newtheorem{lemma}{Lemma}[section]
\theoremstyle{definition}
\newtheorem{definition}{Definition}[section]
\newtheorem{corollary}{Corollary}[section]
\newtheorem{proposition}{Proposition}[section]
\theoremstyle{definition}
\newtheorem{example}{Example}[section]
\theoremstyle{definition}
\newtheorem{exercise}{Exercise}[section]
\theoremstyle{remark}
\newtheorem*{remark}{Remark}
\newtheorem*{solution}{Solution}
\begin{document}
\maketitle

\section{Introduction}\label{introduction}

Phonological transformations can be characterized as mathematical
functions, which map an underlying representation to an attested surface
form. Such transformations can be described using logical transductions
and the expressivity of a logic determines the class of functions that
it describes. For example, previous work has shown that Input Strictly
Local (ISL) string functions are describable with quantifier-free (QF)
first order (FO) logical transductions (Chandlee,
\protect\hyperlink{ref-chandlee2014}{2014}; Chandlee \& Heinz,
\protect\hyperlink{ref-chandleeheinz2018}{2018}). In addition, Chandlee
and Jardine
(\protect\hyperlink{ref-chandleejardineaisl}{2019}\protect\hyperlink{ref-chandleejardineaisl}{a})
showed that some tone processes are QF definable over two-tiered
autosegmental representations (ARs), which make up the Autosegmental
Input Strictly Local (AISL) function class. However, Chandlee and
Jardine
(\protect\hyperlink{ref-chandleejardineaisl}{2019}\protect\hyperlink{ref-chandleejardineaisl}{a})
claimed that unbounded spreading is not QF definable and thus not AISL.
This paper will demonstrate a method for describing vowel harmony
transformations with both unbounded spreading and blocking patterns over
multi-tiered ARs using quantifier-free least fixed point (QFLFP) logical
transductions. This paper will also illustrate that a blocking pattern
is not describable with QFLFP unless the autosegmental association
relation is translated into a function.

First, section 2 defines QF string transductions and the two relations
needed to define autosegmental representations (ARs): ordering and
association. QF transductions are defined using atomic formula that take
terms as a function that maps an input structure to an output copy of
the same type of structure. The structures analyzed in this paper are
multi-tiered autosegmental representations, which utilize ordering
functions between elements on a tier and association between elements on
different tiers.

Section 3 outlines the representational assumptions held throughout this
paper. This paper analyzes two vowel harmony patterns over multi-tiered
ARs, which use a bottlebrush feature representation and binary vowel
features. These ARs abstract away from consonants so they consist of a
vowel tier with elements directly associated to features on separate
tiers. Each feature tier consists of the + and - values of one feature.
Autosegmental theory provides restrictions to allow only output ARs that
obey Full Specification (FS), the No Crossing Constraint (NCC), and the
Obligatory Contour Principle (OCP). Input ARs, on the other hand, are
not subject to FS.

Section 4 introduces the vowel harmony patterns under discussion:
unbounded spreading and blocking. In this section an unbounded spreading
pattern from Akan is defined using QFLFP with a predecessor function and
binary association relation. Spreading is characterized as an iterative
process of associating multiple successive vowels to a single feature
via the recursive lfp predicate. The Akan-like blocking pattern,
however, is not QFLFP definable with a binary association relation.

Section 5 translates the binary association relation into a function.
The association function allows association to be defined using terms,
which adds the possibility for writing a QF predicate that defines
\enquote{unspecified} vowels as elements on the vowel tier, which are
associated to themselves. The recursive QFLFP definition of association
to a particular feature can then can refer to these input reflexive
associations and overwrites them with association to the specified
feature. In this way, the Akan-like blocking pattern is characterized as
iterative spreading of a feature only to vowels that are
\enquote{unspecified}, or associated to themselves, in the input.

Lastly, section 6 concludes and proposes a future extension of this work
to investigate the QFLFP definition of a vowel harmony pattern with
transparent vowels.

\section{QF Logical Transductions}\label{qf-logical-transductions}

QF logic is precisely first-order (FO) logic without quanitfiers
(\(\exists\) or \(\forall\)). In QFFO logic, a term picks out a single
element in a model's domain; (\ref{terms}) lists the mathematical
notation for what is considered a term.

\begin{exe}
\ex Terms $(t)$: \label{terms}
  \begin{xlist}
  \ex a variable $x, y, z, ...$
  \ex a function $f$ applied to a term: $f(t)$
  \end{xlist}
\end{exe}

\noindent QF string transductions define an output string in terms of
the elements and relations of the input string using atomic formulas
that take terms.

\begin{exe}
\ex QF atomic formulas\label{atom}
  \begin{xlist}
  \ex a unary relation: P$(t)$
  \ex a binary relation: R$(t_1, t_2)$ 
  \ex a strict equivalence: $t_1 \approx t_2$
  \end{xlist}
\end{exe}

\noindent For an input alphabet \(\Sigma\) and an output alphabet
\(\Gamma\) a logical transduction consists of a copyset C, a set of
unary output condition predicates, and set of unary predicates to label
the output elements. The copyset contains a copy of each input element,
which is given a label via the unary predicates P\(^C_{\gamma}(x)\). The
unary output condition predicates define the conditions under which the
copy of an input element is present in the output. Lastly, QF
transductions are order-preserving, as defined in (\ref{order.pres}).

\begin{exe}
\ex\label{order.pres} Let $<$ (and$\leq$) be the transitive (and reflexive) closure of p; build p$'$ such that its transitive closure is $<'$, defined as follows: for all $c, e\in C$
\end{exe}

\hspace{1.85in}
\(\begin{array}{ccc} d^c_1 <' d^e_2 \defeq & x<y & if c\geq e \\  & x\leq y & if c<e \end{array}\)
\vspace{0.2in}

\noindent This order preservation allows the first copyset to be a copy
of both the input elements and the ordering between them so that input
ARs are mapped to an output AR.

The following sections will introduce the functions and relations needed
to describe a vowel harmony transformation over multi-tiered
autosegmental representations: order between elements on a tier and
association between elements on different tiers.

\subsection{Ordering with QF}\label{ordering-with-qf}

In order to ensure that QF formulas use terms, which pick out an element
in the model, the binary ordering relation used in previous work with
ARs is changed to a function. Rather than represent ordering between
elements in a string with the binary successor relation (\(\lhd\)) QF
formulas utilize a predecessor function (p(\(x\))) or a successor
function (s(\(x\))). For example, for the string in (\ref{string}) the
predecessor of 5 is 4 and so p(5) = 4, p(p(5)) = 3, etc. In addition,
the successor of 1 is 2, so s(1) = 2, s(s(1)) = 3, etc. Predecessor and
successor must also be total functions, which means they are defined for
every element in the domain so the predicate in (\ref{string}a) defines
the first element as that which is its own predecessor and the predicate
in (\ref{string}b) defines the last element as its own successor.

\begin{exe}
\ex $\langle D; p, s, P_a, P_b, P_c\rangle$ \label{string}\\
\begin{tikzpicture}
\node [circle,draw] (1) at (0,0) {a};
\node [circle,draw] (2) at (1,0) {b};
\node [circle,draw] (3) at (2,0) {b};
\node [circle,draw] (4) at (3,0) {a};
\node [circle,draw] (5) at (4,0) {c};
\draw (1) node [label=below left:1] {};
\draw (2) node [label=below:2] {};
\draw (3) node [label=below:3] {};
\draw (4) node [label=below:4] {};
\draw (5) node [label=below right:5] {};
\path[<-] (1) edge [bend left] node [above] {p} (2);
\path[<-] (2) edge [bend left] node [above] {p} (3);
\path[<-] (3) edge [bend left] node [above] {p} (4);
\path[<-] (4) edge [bend left] node [above] {p} (5);
\path (1) edge [in=190,out=140,loop] node [above] {p} (1);
\path[->] (1) edge [bend right] node [below] {s} (2);
\path[->] (2) edge [bend right] node [below] {s} (3);
\path[->] (3) edge [bend right] node [below] {s} (4);
\path[->] (4) edge [bend right] node [below] {s} (5);
\path (5) edge [in=-10,out=40,loop] node [above] {s} (5);
\end{tikzpicture}

\begin{xlist}
\ex first($x) \defeq$ p($x) \approx x$
\ex last($x) \defeq$ s($x) \approx x$
\end{xlist}
\end{exe}

\noindent The string in (\ref{string}) is defined over a signature that
consists of the predecessor and successor functions and the unary
relation that labels elements with letters of the alphabet: \{p, s,
P\(_{\sigma\in\Sigma}\)\}, \(\Sigma\) = \{a, b, c\}.

ARs consist of more than one string of disjoint sets of elements and
these strings are connected by association. So each tier in a
multi-tiered AR is essentially a string of elements with labels of a
certain type. For example, in the ARs used here the vowel tier consists
of elements labeled with P\(_V(x)\) to represent vowels, while each
feature tier consists of elements labeled with the + or - value of a
particular feature. The predecessor and successor functions are used to
define the order between elements on each tier. The following subsection
discusses the association relation between tiers.

\subsection{Association relation}\label{association-relation}

In addition to ordering between elements on a tier, Chandlee and Jardine
(\protect\hyperlink{ref-chandleejardineaisl}{2019}\protect\hyperlink{ref-chandleejardineaisl}{a})
introduced an antisymmetric binary association relation
\(\mathcal{A}(x, y)\) to describe autosegmental associations between
elements on different tiers. Association was introduced as a binary
relation due to the possible configurations of tone-TBU associations. In
tone languages, it is possible for a single tone to be associated to
multiple TBUs, but in the case of contour tones it is also possible for
a single TBU to be associated to more than one tone. The possibility for
multiple association in either direction rules out the possibility of
representing tone-TBU associations as a function, which must pick out a
single element when applied to another element. In principle the binary
association relation could be defined symmetrically, but I follow
Chandlee and Jardine
(\protect\hyperlink{ref-chandleejardineaisl}{2019}\protect\hyperlink{ref-chandleejardineaisl}{a})
in defining it antisymmetrically. Thus for ARs of tone, \(x\) is always
evaluated as an element on a tonal tier and \(y\) as an element on a
tier of tone-bearing units (TBUs).

This same association relation can be used for multi-tiered ARs of vowel
harmony patterns. In order to do so, a separate association relation
must be defined between the vowel tier and each feature tier. For
example, to describe the transformation in (\ref{assocrel.ex}) the
association between a vowel and a feature on the high tier must be
defined separately from the association between a vowel and a feature on
the back tier. Both associations are defined for the output in terms of
the input.

\begin{exe}
\ex Separate association relations for each feature tier\label{assocrel.ex}\\
\begin{multicols}{2}
\begin{tikzpicture}[baseline=(current bounding box.north)]
 \matrix [matrix of nodes,
          row sep=1em, 
          nodes={text height=1em, text depth=0.5em}] {
|(h)| + high &           & |(3)| +high\\
|(v)| V      & $\mapsto$ & |(1)| V \\
|(b)| - back &           & |(4)| +back\\
  };
 \draw (h.south) -- (v.north);
 \draw (v.south) -- (b.north);
 \draw (3.south) -- (1.north);
 \draw (1.south) -- (4.north);
\end{tikzpicture}

\columnbreak
\begin{xlist}
  \ex $\mathcal{A}_{high}'(x, y) \defeq \mathcal{A}_{high}(x, y)$
  \ex $\mathcal{A}_{back}'(x, y) \defeq \mathcal{A}_{back}(x, y)$
\end{xlist}
\end{multicols}
\end{exe}

\noindent Since both output featural associations are also present in
the input, the QF definitions of both the input and output associations
are identical. The output association relation is denoted with \('\).

This section established the mathematical definitions that will be used
to describe vowel harmony transformations as QF logical transductions.
It introduced QF logic as well as the ordering functions and binary
association relation needed to define multi-tiered ARs using QF logic.
However, on its own QF is not sufficient to describe the vowel harmony
transformations discussed in this paper. The definitions presented here
along with the discussion in the next section provide the basis for
understanding the logical descriptions of both unbounded spreading and
blocking transformations using QFLFP discussed in sections 4 and 5.

\section{Representational
Assumptions}\label{representational-assumptions}

This paper explores the use of Quantifier-free Least Fixed Point logic
(QFLFP) for describing vowel harmony transformations over multi-tiered
autosegmental representations (ARs). Autosegmental representations (ARs)
of tonal patterns generally consist of two tiers: the TBU and segmental
tiers (Goldsmith, \protect\hyperlink{ref-Goldsmith1976}{1976}; Jardine,
\protect\hyperlink{ref-jardinediss}{2016},
\protect\hyperlink{ref-jardinelocaltone}{2017}), but vowel harmony
patterns refer to subsegmental features, which will be represented using
multiple featural tiers. The multi-tiered ARs in this paper utilize a
version of Hayes (\protect\hyperlink{ref-hayes1990}{1990})'s bottlebrush
theory of vowel feature representations such that each feature is
represented on a separate tier. Assuming binary vowel features means
that each feature tier includes both the + and - values of a particular
feature. Associations connect the segmental and feature tiers such that
each vowel is directly associated to an element on multiple feature
tiers. The ARs used here abstract away from consonants under the
assumption that they do not participate in vowel harmony and thus are
not associated to vowel features; the \enquote{spine of the
bottlebrush}(Hayes, \protect\hyperlink{ref-hayes1990}{1990}) only
includes vowels. Association relations are represented by straight lines
that connect elements (segments and features) on different tiers. Where
a tier consists of multiple elements, the successor ordering relation
between elements on that tier is represented with horizontal arrows. An
example is shown in (\ref{ar.ex}).

\begin{exe}
\ex Multi-tiered AR \label{ar.ex}\\
\begin{tikzpicture}[baseline=(current bounding box.north)]
  \matrix [matrix of nodes, ampersand replacement=\&, row sep=1.25ex, column sep=2.25ex, nodes={text height=0.75em, text depth=0.25em}] 
  {
               \& |(b)| $\pm$back \\
|(a)| $\pm$ATR \&                 \\
|(1)| V        \&                 \\
|(l)| $\pm$low \&                 \\
               \& |(r)| $\pm$round \\
  };
  \draw (a.south) -- (1.north);
  \draw (b.south) -- (1.north);
  \draw (1.south) -- (l.north);
  \draw (1.south) -- (r.north);
\end{tikzpicture}
\end{exe}

Developing logical descriptions of vowel harmony transformations
provides insights into the necessity of certain restrictions on both
input (underlying) and output (surface) structures. Use of ARs requires
discussion of at least some of the basic representational assumptions
that will be investigated in this paper. This paper assumes \emph{Full
Specification} (FS)(Clements,
\protect\hyperlink{ref-Clements1976}{1976}) for surface, but not
underlying representations. The \emph{Obligatory Contour Principle}
(OCP)(Leben, \protect\hyperlink{ref-leben1973}{1973}) and the \emph{No
Crossing Constraint} (NCC) (Goldsmith,
\protect\hyperlink{ref-Goldsmith1976}{1976}; Sagey,
\protect\hyperlink{ref-sagey1986}{1986}) are assumed for surface
representations of spreading. Examples of structures that violate each
of these constraints are shown in (\ref{fs.ex})-(\ref{ocp.ex}) below.

\begin{exe}
\ex \label{fs.ex} Violates FS\\
  \begin{tikzpicture}[baseline=(current bounding box.north)]
  \matrix [matrix of nodes, row sep=2.5ex, column sep=2.25ex, nodes={text height=1em, text depth=0.5em}] 
  {
* & |(b)| -ATR & \\
  & |(d)| V    & |(e)| V \\
  & |(f)| -low & \\
  };
  \draw (b.south) -- (d.north);
  \draw (d.south) -- (f.north);
  \draw[thick,black,->] (d) -- (e);
  \draw foreach \x in {d, e} {(\x.south) -- (f.north)};
  \end{tikzpicture}
\end{exe}

\begin{exe}
\ex \label{ncc.ex} Violates NCC\\
  \begin{tikzpicture}[baseline=(current bounding box.north)]
  \matrix [matrix of nodes, row sep=2.5ex, column sep=2.25ex, nodes={text height=1em, text depth=0.5em}] 
  {
* & |(a)| +ATR & |(b)| -ATR \\
  & |(c)| V    & |(d)| V \\
  & |(e)| -low &  \\
  };
  \draw (a.south) -- (d.north);
  \draw (b.south) -- (c.north);
  \draw[thick,black,->] (a) -- (b);
  \draw[thick,black,->] (c) -- (d);
  \draw foreach \x in {c, d} {(\x.south) -- (e.north)};
  \end{tikzpicture}
\end{exe}

\begin{exe}
\ex \label{ocp.ex} Violates OCP\\
  \begin{tikzpicture}[baseline=(current bounding box.north)]
  \matrix [matrix of nodes, row sep=2.5ex, column sep=2.25ex, nodes={text height=1em, text depth=0.5em}] 
  {
* & |(a)| -ATR & |(b)| -ATR \\
  & |(c)| V    & |(d)| V \\
  & |(e)| -low & |(f)| -low \\
  };
  \draw (a.south) -- (c.north);
  \draw (b.south) -- (d.north);
  \draw (c.south) -- (e.north);
  \draw (d.south) -- (f.north);
  \draw[thick,black,->] (a) -- (b);
  \draw[thick,black,->] (c) -- (d);
  \draw[thick,black,->] (e) -- (f);
  \end{tikzpicture}
\end{exe}

First, assuming FS for surface structures means that each output element
on a feature tier must be associated to at least one output element on
the vowel tier and each output element on the vowel tier must be
associated to at least one output element on each featural tier. FS
crucially allows vowels to be associated to multiple featural tiers as
is necessary for each vowel feature to occupy its own tier. The
hypothetical AR in (\ref{fs.ex}) straighforwardly violates FS because
there is a vowel that is not associated to any feature on the ATR tier.
Since this paper examines vowel harmony patterns I assume that
consonants can not to be associated to vowel features and so FS and
vowel harmony in general ignore consonants.

Second, the NCC states that association lines between the segmental tier
and a feature tier never cross. Odden
(\protect\hyperlink{ref-odden1994}{1994}) adds that the NCC can only
evaluate the association between the vowel and one featural tier at a
time. The representation in (\ref{ncc.ex}) violates the NCC because +ATR
precedes -ATR, but is associated to a vowel that is preceded by a vowel
associated to -ATR; this configuration creates visually crossed
association lines.

A notable effect of FS along with the NCC on surface representations is
that they prevent what have been called gapped structures (Archangeli \&
Pulleyblank, \protect\hyperlink{ref-archangelipulleyblank1994}{1994};
Ringen \& Vago, \protect\hyperlink{ref-ringenvago1998}{1998}). A gapped
structure is one in which a feature appears to have skipped over a vowel
that it could potentially be associated to. FS would prevent gapped
structures in which the \enquote{skipped} vowel is not associated to
anything on that particular feature's tier. The NCC would prevent gapped
structures in which the surrounding two vowels are associated to a
single feature and the intervening \enquote{skipped} vowel is associated
to a different feature on the same tier.

Lastly, the OCP stipulates that successive featural elements must be
distinct. The representation in (\ref{ocp.ex}) violates the OCP because
on both the ATR and low feature tiers there are two identical successive
features, -ATR and -low respectively. The OCP in conjunction with FS
results in representations in which multiple vowels are associated to a
single feature rather than having multiple successive iterations of the
same feature value on a tier each associated to a single vowel. An
example representation of an Akan word that satisfies all of the AR
properties discussed here is shown in (\ref{akan.ex}).

On the surface, both the NCC and the OCP have also been derived via a
concatenation operation (\(\circ\)) that merges autosegmental
\enquote{graph primitives}(Jardine \& Heinz,
\protect\hyperlink{ref-jardineheinz2015}{2015}, p. 1). An autosegmental
graph primitive consists of an element on the vowel tier, the elements
on each feature tier and the associations between the feature and vowel
tiers. The concatenation operation combines a finite set of adjacent
graph primitives to generate a fully specified AR. For example, the AR
in (\ref{akan.ex}) is derived from the set of graph primitives in
(\ref{concat.ex}). Each primitive in (\ref{concat.ex}) is concatenated
with a single adjacent primitive. If two adjacent primitives share an
identical feature those two features are merged into one feature with
two associations, as in (\ref{akan.ex}). The merging of identical
adjacent features essentially prevents surface ARs from having multiple
iterations of a feature and crossed associations, thus satisfying both
the OCP and the NCC. However, if two segmental elements are associated
to the exact same feature and a different element intervenes then both
iterations of that feature will occur in the surface AR because only
adjacent primitive elements are concatenated and can thus be merged. An
intervening element can be a vowel associated to the same feature with a
different value.

\begin{exe}
\ex \label{concat.ex} Concatenation of adjacent autosegmental graph primitives \\
  \begin{tikzpicture}[baseline=(current bounding box.north)]
  \matrix [matrix of nodes, row sep=2.5ex, column sep=2.25ex, nodes={text height=1em, text depth=0.5em}] 
  {
  &                                     & |(a)| -ATR &                                     & |(b)| -ATR \\
t & \node {}; \draw(0, 0) circle (3pt); & |(1)| i    & \node {}; \draw(0, 0) circle (3pt); & |(2)| e \\
  &                                     & |(c)| -low &                                     & |(d)| -low \\
  };
  \draw (a.south) -- (1.north);
  \draw (1.south) -- (c.north);
  \draw (b.south) -- (2.north);
  \draw (2.south) -- (d.north);
  \end{tikzpicture}
\end{exe}

\begin{exe}
\ex \label{akan.ex} Surface AR satisfies FS, NCC, and OCP \\
  \begin{tikzpicture}[baseline=(current bounding box.north)]
  \matrix [matrix of nodes, row sep=2.5ex, column sep=2.25ex, nodes={text height=1em, text depth=0.5em}] 
  {
  &            & |(b)| -ATR &         & &           & &            & |(b2)| -ATR & \\
  & |(c)| t    & |(d)| i    & |(e)| e & & $\mapsto$ & & |(c2)| t   & |(d2)| i    & |(e2)| e \\
  &            & |(f)| -low &         & &           & &            & |(f2)| -low & \\
  };
  \draw (b.south) -- (d.north);
  \draw foreach \x in {d, e} {(\x.south) -- (f.north)};
  \draw[thick,black,->] (c) -- (d);
  \draw[thick,black,->] (d) -- (e);
  \draw foreach \x in {d2, e2} {(b2.south) -- (\x.north)};
  \draw foreach \x in {d2, e2} {(\x.south) -- (f2.north)};
  \draw[thick,black,->] (c2) -- (d2);
  \draw[thick,black,->] (d2) -- (e2);
  \end{tikzpicture}
\end{exe}

In the transformation in (\ref{akan.ex}), the initial consonant cannot
be associated to a vowel feature in either the underlying AR (on the
left) or the surface AR (on the right). While the consonant is ordered
with respect to the vowels, FS does not require the consonant to be
associated to any element on either feature tier. FS also does not apply
to the underlying representation, in which the second vowel is not
associated to any ATR feature. On the surface, \emph{tie} satisfies FS
because the ATR feature has spread so that each vowel is associated to a
feature on each of the featural tiers and all features are associated to
at least one vowel. Both the underlying and surface ARs of \emph{tie}
also satisfy both the NCC and the OCP because there is only one of each
feature. The features are represented on separate tiers so association
lines cannot cross and there is nothing else on those tiers that could
violate the OCP.

In this section I have outlined representational assumptions that will
be utilized throughout this paper. These assumptions are adapted from
the autosegmental literature and from my previous work with multi-tiered
ARs of vowel harmony. The multi-tiered ARs used here assume a
bottlebrush feature representation such that one tier consists of only
vowels, which are each associated to features on different tiers. Each
feature tier consists only of the + and - values of one feature. Surface
ARs are assumed to obey FS, the NCC, and the OCP, but underlying ARs are
subject only to the NCC and the OCP. Future extensions of this paper may
further investigate the necessity of these assumptions when extending
QFLFP to describe vowel harmony transformations with transparent vowels.

\section{Vowel harmony patterns}\label{vowel-harmony-patterns}

Vowel harmony is considered a process of assimilation that affects only
the vowels in a word. Different theories suggest different mechanisms
for vowel harmony assimilation, but this paper focuses on the logical
description of transformations and seeks only to probe their
computational complexity. In this paper computational complexity is
investigated via the expressive power of the logic used to describe
vowel harmony transformations. This section will discuss the logical
description of two transformations which form the basis of many attested
vowel harmony patterns: unbounded spreading and blocking.

In order to describe these vowel harmony patterns the QF FO logic
defined in section 2 is supplemented with a least fixed point (lfp)
operator (Libkin, \protect\hyperlink{ref-libkin2012}{2012}) to describe
feature spreading. This operator applies to the set of things denoted by
the binary variable, which allows a formula in the definition for an
output association, for example, to iteratively apply associations
following a single underlying association. The lfp operator introduces a
binary set variable, which takes two arguments. Using the lfp operator
with QF logic requires that a formula within the scope of the lfp has at
most two free variables. This supplemented QF logic is called QFLFP.

\subsection{Unbounded spreading}\label{unbounded-spreading}

A transformation in which no elements, orders, or associations change
from the input to the output can be easily described using QF logic, but
few vowel harmony patterns work this way. One pattern that is commonly
discussed in autosegmental literature is unbounded spreading. Unbounded
spreading occurs in languages with full harmony and in languages where
some words undergo full harmony and others do not, like Akan. In vowel
harmony terms, an unbounded rightward spreading pattern consists of an
input with some number of vowels and only the first one is associated to
a feature. That feature then spreads so that the output structure is
fully specified. In other words, the output structure consists of the
same number of vowels all associated to that same single feature. In
Akan, the feature that spreads in fully harmonic words is either
{[}+ATR{]} or {[}-ATR{]} and ATR spreading interacts with low vowels,
but this interaction will not be discussed until section 4.2. The ARs in
(\ref{unbound.spread}) illustrate unbounded rightward spread of +ATR
among -low vowels.

\begin{exe}
\ex \label{unbound.spread} Unbounded ATR Spreading \\
  \begin{tikzpicture}[baseline=(current bounding box.north)]
  \matrix [matrix of nodes, row sep=2.5ex, column sep=2.25ex, nodes={text height=1em, text depth=0.5em}] 
  {
|(a)| +ATR$_1$ &             &             &           & |(c)| +ATR$_{1'}$\\
|(1)| V$_2$    & |(2)| V$_3$ & |(3)| V$_4$ & $\mapsto$ & |(4)| V$_{2'}$  & |(5)| V$_{3'}$ & |(6)| V$_{4'}$\\
|(b)| -low$_5$ &             &             &           & |(d)| -low$_{5'}$\\
  };
  \draw (a.south) -- (1.north);
  \draw foreach \x in {1, 2, 3} {(\x.south) -- (b.north)};
  \draw foreach \x in {4, 5, 6} {(c.south) -- (\x.north)};
  \draw foreach \x in {4, 5, 6} {(\x.south) -- (d.north)};
  \draw[black,<-] (1) -- (2);
  \draw[black,<-] (2) -- (3);
  \draw[black,<-] (4) -- (5);
  \draw[black,<-] (5) -- (6);
  \end{tikzpicture}
\end{exe}

In (\ref{unbound.spread}) each element is represented with a subscript
number, the labels are represented either by the feature's name and
value or by a V, for vowels. The elements in the output AR (denoted with
\('\)) are all also present in the input AR and so the label they
receive can be easily defined using QF logic, as in
(\ref{dunbound.def}). The first definition (P\('_{D}(x)\)) specifies
that each domain element in the input is present in the output. The
other three definitions label each output element with its input label.

\begin{exe}
\ex\label{dunbound.def} $\langle$D; p, $\mathcal{A}$, P$_V$, P$_{+ATR}$, P$_{-ATR}$, P$_{+low}$, P$_{-low}\rangle$ 
\end{exe}

P\('_{D}(x) \defeq\) P\(_{D}(x)\) \hspace{1.2in} P\('_{V}(x) \defeq\)
P\(_{V}(x)\)

P\('_{+ATR}(x) \defeq\) P\(_{+ATR}(x)\) \hspace{0.75in}
P\('_{-low}(x) \defeq\) P\(_{-low}(x)\) \vspace{0.25in}

\noindent Similarly, the output associations between vowels and the
{[}-low{]} feature are all present in the input as well, so I can write
the definition for association between vowels and low features as in
(\ref{lowunbound.def}).

\begin{exe}
\ex\label{lowunbound.def}
$\mathcal{A}_{low}'(x, y) \defeq \mathcal{A}_{low}(x, y)$
\end{exe}

However, the current QF version of FO logic does not provide a way to
define the spreading of a {[}+ATR{]} feature to an unbounded number of
vowels, as in (\ref{unbound.spread}). Chandlee and Jardine
(\protect\hyperlink{ref-chandleejardineaisl}{2019}\protect\hyperlink{ref-chandleejardineaisl}{a})
argue that unbounded tone spread is not A-ISL because the output
association is not QF definable. In order to define an unboundedly
spreading association, a least-fixed point (lfp) operator must be
introduced (Libkin, \protect\hyperlink{ref-libkin2012}{2012}). The lfp
operator introduces a recursive binary variable, which allows the
evaluation of the association relation to apply iteratively over
successive vowels. In this way, unbounded feature spreading can be
defined as the iterative association of a feature from one underlying
vowel onto all successive vowels. The unbounded spreading of +ATR from
left to right in (\ref{unbound.spread}) can thus be defined using
quantifier-free least fixed point logic (QFLFP), as in
(\ref{unboundqflfp.def}).

\begin{exe}
\ex\label{unboundqflfp.def}
$\mathcal{A}_{ATR}'(x, y) \defeq [lfp \mathcal{A}_{ATR}(x, y)\vee R(x', p(y'))](x, y)$
\end{exe}

\noindent QFLFP requires a maximum of two free variables so the lfp
operation requires reference to only two elements: an element on the
relevant feature tier and an element on the vowel tier. As mentioned
above, the \(x\) is evaluated as a featural element and \(y\) as a
vocalic element; so according to the formula in (\ref{unboundqflfp.def})
an output association will be drawn between the {[}+ATR{]} feature and a
vowel if that vowel is underlyingly associated to the {[}+ATR{]} feature
or if the preceding vowel has been associated to that feature.

\subsection{Blocking}\label{blocking}

In Akan, a {[}+low{]} vowel blocks the spread of ATR. In autosegmental
terms, this means that an underlying {[}+ATR{]} feature, for example,
spreads only to {[}-low{]} vowels and does not associate to a {[}+low{]}
vowel. The multi-tiered ARs used throughout this paper do not provide a
straightforward way to illustrate such an interaction between feature
tiers unless the {[}+low{]} vowel is also associated to {[}-ATR{]} in
the underlying form. The lfp operator allows a feature to spread its
underlying association from one vowel to multiple vowels, but if an
underlyingly {[}+low, -ATR{]} vowel blocks the spread of {[}+ATR{]} the
transformation will look like (\ref{block}).

\begin{exe}
  \ex \label{block} Blocking\\
  \begin{tikzpicture}[baseline=(current bounding box.north)]
  \matrix [matrix of nodes, row sep=2.5ex, column sep=2.25ex, nodes={text height=1em, text depth=0.5em}] 
  {
|(a1)| +ATR$_1$ &         & |(b1)| -ATR$_2$ &         &           & |(c1)| +ATR$_{1'}$ &         & |(d1)| -ATR$_{2'}$\\
|(1)| V$_3$  & |(2)| V$_4$ & |(3)| V$_5$      & |(4)| V$_6$ & $\mapsto$ & |(5)| V$_{3'}$     & |(6)| V$_{4'}$ & |(7)| V$_{5'}$ & |(8)| V$_{6'}$\\
|(a2)| -low$_7$ &         & |(b2)| +low$_8$ &         &           & |(c2)| -low$_{7'}$ &         & |(d2)| +low$_{8'}$\\
  };
  \draw (a1.south) -- (1.north);
  \draw foreach \x in {1, 2} {(\x.south) -- (a2.north)};
  \draw (b1.south) -- (3.north);
  \draw foreach \x in {3, 4} {(\x.south) -- (b2.north)};
  \draw foreach \x in {5, 6} {(c1.south) -- (\x.north)};
  \draw foreach \x in {5, 6} {(\x.south) -- (c2.north)};
  \draw foreach \x in {7, 8} {(d1.south) -- (\x.north)};
  \draw foreach \x in {7, 8} {(\x.south) -- (d2.north)};
  \draw[black,<-] (a1) -- (b1);
  \draw[black,<-] (c1) -- (d1);
  \draw[black,<-] (1) -- (2);
  \draw[black,<-] (2) -- (3);
  \draw[black,<-] (3) -- (4);
  \draw[black,<-] (5) -- (6);
  \draw[black,<-] (6) -- (7);
  \draw[black,<-] (7) -- (8);
  \draw[black,<-] (a2) -- (b2);
  \draw[black,<-] (c2) -- (d2);
  \end{tikzpicture}
\end{exe}

As before, each element in the domain is labeled by defining the unary
output predicates and the same number of elements are present in the
input as are in the output. However, in order to describe when spreading
is blocked the formula must refer either to the second featural element
on the ATR tier labeled with {[}-ATR{]} and associated to a vowel in the
input (2) or to vowels that are not associated in the input AR (4 and
6). A QFLFP definition of the transformation in (\ref{block}) is
attempted in (\ref{qflfp.block}).

\begin{exe}
\ex\label{qflfp.block} $\langle$D; p, s, $\mathcal{A}$, P$_V$, P$_{+ATR}$, P$_{-ATR}$, P$_{+low}$, P$_{-low}\rangle$ 
\end{exe}

P\('_{D}(x) \defeq\) P\(_{D}(x)\) \hspace{1.07in} P\('_{V}(x) \defeq\)
P\(_{V}(x)\)

P\('_{+ATR}(x) \defeq\) P\(_{+ATR}(x)\) \hspace{0.65in}
P\('_{-ATR}(x) \defeq\) P\(_{+ATR}(x)\)

P\('_{+low}(x) \defeq\) P\(_{+low}(x)\) \hspace{0.75in}
P\('_{-low}(x) \defeq\) P\(_{+low}(x)\)

\(\mathcal{A}_{ATR}'(x, y) \defeq [lfp \mathcal{A}_{ATR}(x, y)\vee R(x', p(y'))\wedge ...]\)

\(\mathcal{A}_{low}'(x, y) \defeq [lfp \mathcal{A}_{low}(x, y)\vee R(x', p(y'))\wedge ...]\)
\vspace{0.25in}

\noindent In order to refer to a second element on the ATR feature tier,
the formula must introduce a third free variable, but the binary set
variable introduced by the lfp operator only allows for two. In
addition, the binary association relation provides a means by which to
refer to elements on different tiers that are associated to one another,
but there is not way to refer to unassociated elements. The alternative
discussed in the next section involves drawing an analogy between the
association and ordering relations and once again changing the model
signature over which multi-tiered ARs are defined in QFLFP.

In this section I have shown that an unbounded spreading transformation
over multi-tiered ARs can be described using QFLFP and a binary
association relation, but not a blocking transformation. Unbounded
spreading QFLFP definable because the lfp operator introduces a binary
set variable, which allows exactly two variables to be free within the
scope of the lfp predicate. The binary set variable takes two terms,
which allows the output association to be recursively defined. In other
words, the QFLFP definition of spreading is an iterative process of
association from successive vowels to a single feature. However, a
blocking transformation is not QFLFP definable using the binary
association relation because the lfp predicate can have only two free
variables (\(x\) and \(y\)) and there is no way to refer to a second
input feature without introducing a third variable. In addition, there
is no way to refer to unassociated vowels so the lfp predicate cannot
specify onto which vowels the feature spreads.

\section{Association Function}\label{association-function}

Unlike with tone, vowel harmony does not allow a vowel to be associated
to more than one element on a feature tier. In other words, vowel
harmony patterns do not include ARs in which a single vowel is
associated to multiple elements on a feature tier; multiple association
can only be unidirectional. As a result, the binary autosegmental
association relation introduced by Chandlee and Jardine
(\protect\hyperlink{ref-chandleejardineaisl}{2019}\protect\hyperlink{ref-chandleejardineaisl}{a})
could be reformulated as a function (\(\alpha\)), which associates
elements on the vowel tier to an element on a feature tier. The analogy
could be drawn here between association and ordering because the binary
ordering relations were also changed to functions when moving from FO to
QFFO logic. Again, a separate association function must be defined for
each feature tier. Since featural associations are all the same type of
function (\(\alpha_F(x)\approx y\)) they will work in the same way. In
order to evaluate \(\alpha_F(x)\approx y\), the variable \(x\) is
evaluated as an element on the vowel tier and \(y\) as an element on a
feature tier. The association function will be represented with arrows
from a vowel to a feature, as in (\ref{block.func}).

\subsection{Unbounded Spreading}\label{unbounded-spreading-1}

The Akan patterns discussed in the previous section will now be
reanalyzed using the new association function. First, the same unbounded
spreading pattern as in (\ref{unbound.spread}) is reproduced here in
(\ref{block.func}) using the association functions, \(\alpha_{ATR}\) and
\(\alpha_{low}\).

\begin{exe}
\ex \label{block.func} Association as a unidirectional function \\
  \begin{tikzpicture}[baseline=(current bounding box.north)]
  \matrix [matrix of nodes, row sep=2.5ex, column sep=2.25ex, nodes={text height=1em, text depth=0.5em}] 
  {
|(a)| +ATR$_1$ &             &             &           & |(c)| +ATR$_{1^1}$\\
|(1)| V$_2$    & |(2)| V$_3$ & |(3)| V$_4$ & $\mapsto$ & |(4)| V$_{2^1}$ & |(5)| V$_{3^1}$ & |(6)| V$_{4^1}$ \\
|(b)| -low$_5$ &             &             &           & |(d)| -low$_{5^1}$\\
  };
  \draw[black,<-] (a.south) -- (1.north);
  \draw[black,->] (1.south) -- (b.north);
  \draw[black,->] (2.south) -- (b.north);
  \draw[black,->] (3.south) -- (b.north);
  \draw[black,<-] (c.south) -- (4.north);
  \draw[black,<-] (c.south) -- (5.north);
  \draw[black,<-] (c.south) -- (6.north);
  \draw[black,->] (4.south) -- (d.north);
  \draw[black,->] (5.south) -- (d.north);
  \draw[black,->] (6.south) -- (d.north);
  \draw[black,<-] (1) -- (2);
  \draw[black,<-] (2) -- (3);
  \draw[black,<-] (4) -- (5);
  \draw[black,<-] (5) -- (6);
  \end{tikzpicture}
\end{exe}

\noindent Again, each element in the output also appears in the input so
the unary predicates can be easily defined with QF, as in
(\ref{fblock.def}). The output functions in (\ref{blockfunc.def}) are
defined in terms of the input associations by using the lfp operator and
binary variable introduced in the previous section, which allows the
formula to have exactly two free variables.

\newpage

\begin{exe}
\ex\label{fblock.def} $\langle$D; p, $\alpha$, P$_V$, P$_{+ATR}$, P$_{-low}\rangle$ 
\end{exe}

P\('_{D}(x) \defeq\) P\(_{D}(x)\) \hspace{1.2in} P\('_{V}(x) \defeq\)
P\(_{V}(x)\)

P\('_{+ATR}(x) \defeq\) P\(_{+ATR}(x)\) \hspace{0.75in}
P\('_{-low}(x) \defeq\) P\(_{-low}(x)\) \vspace{0.25in}

\begin{exe}
\ex \label{blockfunc.def}
  \begin{xlist}
  \ex $\alpha_{ATR}'(x)\approx y \defeq$ [lfp $\alpha_{ATR}(x)\approx y\vee R(p(x'), y')]$$(x,y)$
  \ex $\alpha_{low}'(x)\approx y \defeq \alpha_{low}(x)\approx y$
  \end{xlist}
\end{exe}

\noindent Unbounded ATR spreading is thus described using the
association function as in (\ref{blockfunc.def}a), which states that an
element on the vowel tier is associated to an element on the ATR tier in
the output copy if it is associated to that element on the ATR tier in
the input or if its predecessor is associated to the same element on the
ATR tier in the output. This description is true for 2', 3', and 4',
which are all associated to 1'.

The analogy with ordering does not end at making association a function,
though. Section 2.1 explains that the ordering functions are necessarily
total functions, which means they are by defined for each element in the
domain. This is made possible by defining external predicates, which
define the first element in a string as its own predecessor and the last
element in a string as its own successor. In the same way, the
association function is defined for all elements on the vowel tier such
that underlyingly unassociated vowels are associated to themselves.
These \enquote{unspecified} vowels are also defined with an external
predicate for each association function, as in (\ref{unassoc.pred}).

\begin{exe}
\ex \label{unassoc.pred}
  \begin{xlist}
  \ex unspec$_{ATR}(x) \defeq \alpha_{ATR}(x)\approx x$ 
  \ex unspec$_{low}(x) \defeq \alpha_{low}(x)\approx x$
  \end{xlist}
\end{exe}

\noindent The final transformation for an unbounded rightward ATR
spreading pattern thus looks like (\ref{unassoc.ars}). The unary
predicates are defined logically as in (\ref{unassoc.def}) and
association functions are defined in (\ref{unassoc.def2}).

\begin{exe}
\ex \label{unassoc.ars} Association as a total function \\
  \begin{tikzpicture}[baseline=(current bounding box.north)]
  \matrix [matrix of nodes, row sep=2.5ex, column sep=2.25ex, nodes={text height=1em, text depth=0.5em}] 
  {
|(a)| +ATR$_1$ &             &             &           & |(c)| +ATR$_{1'}$\\
|(1)| V$_2$    & |(2)| V$_3$ & |(3)| V$_4$ & $\mapsto$ & |(4)| V$_{2'}$ & |(5)| V$_{3'}$ & |(6)| V$_{4'}$ \\
|(b)| -low$_5$ &             &             &           & |(d)| -low$_{5'}$\\
  };
  \draw[black,<-] (a.south) -- (1.north);
  \draw[black,->] (1.south) -- (b.north);
  \draw[black,->] (2.south) -- (b.north);
  \draw[black,->] (3.south) -- (b.north);
  \draw[black,<-] (c.south) -- (4.north);
  \draw[black,<-] (c.south) -- (5.north);
  \draw[black,<-] (c.south) -- (6.north);
  \draw[black,->] (4.south) -- (d.north);
  \draw[black,->] (5.south) -- (d.north);
  \draw[black,->] (6.south) -- (d.north);
  \path (2) edge [loop above] (2);
  \path (3) edge [loop above] (3);
  \draw[black,<-] (1) -- (2);
  \draw[black,<-] (2) -- (3);
  \draw[black,<-] (4) -- (5);
  \draw[black,<-] (5) -- (6);
  \end{tikzpicture}

\ex $\langle$D; p, $\alpha$, P$_V$, P$_{+ATR}$, P$_{-low}\rangle$\label{unassoc.def} 
\end{exe}

P\('_{D}(x) \defeq\) P\(_{D}(x)\) \hspace{1.07in} P\('_{V}(x) \defeq\)
P\(_{V}(x)\)

P\('_{+ATR}(x) \defeq\) P\(_{+ATR}(x)\) \hspace{0.65in}
P\('_{-ATR}(x) \defeq\) P\(_{-ATR}(x)\)

P\('_{+low}(x) \defeq\) P\(_{+low}(x)\) \hspace{0.75in}
P\('_{-low}(x) \defeq\) P\(_{-low}(x)\)

\newpage

\begin{exe}
\ex \label{unassoc.def2}
  \begin{xlist}
  \ex $\alpha_{ATR}'(x)\approx y \defeq [lfp \alpha_{ATR}(x)\approx y\vee R(p(x'), y')](x,y)$ 
  \ex $\alpha_{low}'(x)\approx y \defeq \alpha_{low}(x)\approx y](x,y)$
  \end{xlist}
\end{exe}

\noindent The definition of (\(\alpha_{ATR}'(x)\approx y\)) effectively
overwrites any input associations that are not included in the lfp
formula in (\ref{unassoc.def2}a), but the unspec\(_F(x)\) predicates
will be useful in the next subsection.

\subsection{Blocking}\label{blocking-1}

I proposed the association function as an alternative to the binary
association relation, which could not describe a feature spreading
pattern with blocking, like Akan. In the previous section I explained
that the formula which defines the output ATR associations must refer to
either a second ATR feature or the underlyingly unspecified vowels. The
association function and unspec predicates allow a QFLFP formula to
refer directly to elements on the vowel tier which are both associated
and unassociated to an ATR feature. In this way, QFLFP can be used to
describe the blocking pattern of Akan if the blocking {[}+low{]} vowel
is also underlyingly associated to {[}-ATR{]}.

\begin{exe}
  \ex \label{block.unspec} Blocking\\
  \begin{tikzpicture}[baseline=(current bounding box.north)]
  \matrix [matrix of nodes, row sep=2.5ex, column sep=2.25ex, nodes={text height=1em, text depth=0.5em}] 
  {
|(a1)| +ATR$_1$ &         & |(b1)| -ATR$_2$ &         &           & |(c1)| +ATR$_{1^1}$ &         & |(d1)| -ATR$_{2^1}$\\
|(1)| V$_3$ & |(2)| V$_4$ & |(3)| V$_5$ & |(4)| V$_6$ & $\mapsto$ & |(5)| V$_{3^1}$ & |(6)| V$_{4^1}$ & |(7)| V$_{5^1}$ & |(8)| V$_{6^1}$\\
|(a2)| -low$_7$ &         & |(b2)| +low$_8$ &         &           & |(c2)| -low$_{7^1}$ &         & |(d2)| +low$_{8^1}$\\
  };
  \draw (a1.south) -- (1.north);
  \draw foreach \x in {1, 2} {(\x.south) -- (a2.north)};
  \draw (b1.south) -- (3.north);
  \draw foreach \x in {3, 4} {(\x.south) -- (b2.north)};
  \draw foreach \x in {5, 6} {(c1.south) -- (\x.north)};
  \draw foreach \x in {5, 6} {(\x.south) -- (c2.north)};
  \draw foreach \x in {7, 8} {(d1.south) -- (\x.north)};
  \draw foreach \x in {7, 8} {(\x.south) -- (d2.north)};
  \path (2) edge [loop above] (2);
  \path (4) edge [loop above] (4);
  \draw[black,<-] (1) -- (2);
  \draw[black,<-] (2) -- (3);
  \draw[black,<-] (3) -- (4);
  \draw[black,<-] (5) -- (6);
  \draw[black,<-] (6) -- (7);
  \draw[black,<-] (7) -- (8);
  \end{tikzpicture}
\end{exe}

\noindent For this reason, the input AR in (\ref{block.unspec}) has two
underlying ATR associations. In the input 3 associated to 7 and 1, and
the {[}+ATR{]} feature spreads so that 4 is also associated to 1 in the
output. However, 5 is associated to 2 and 8 and so it cannot also be
associated to 1, thus blocking the spread of {[}+ATR{]}.

The definitions of the blocking transformation in (\ref{block.unspec})
look similar to the definitions of the unbounded spreading
transformation in (\ref{unassoc.ars}), except for the output condition
on ATR association. All of the input elements are copied into the output
so the output conditions for the unary predicates are easily defined
using QF, as in (\ref{blockunspec.def1}).

\begin{exe}
  \ex $\langle$D; p, $\alpha$, P$_V$, P$_{+ATR}$, P$_{-ATR}$, P$_{+low}$, P$_{-low}\rangle$\label{blockunspec.def1}
  \begin{xlist}
  \ex P$'_{D}(x) \defeq$ P$_{D}(x)$ \hspace{1.07in} P$'_{V}(x) \defeq$ P$_{V}(x)$
  \ex P$'_{+ATR}(x) \defeq$ P$_{+ATR}(x)$ \hspace{0.65in} P$'_{-ATR}(x) \defeq$ P$_{-ATR}(x)$
  \ex P$'_{+low}(x) \defeq$ P$_{+low}(x)$ \hspace{0.75in} P$'_{-low}(x) \defeq$ P$_{-low}(x)$
  \end{xlist}
  
\ex \label{blockunspec.def2}
  \begin{xlist}
  \ex $\alpha_{ATR}'(x)\approx y \defeq [lfp(\alpha_{ATR}(x)\approx y\wedge \neg x\approx y) \vee (R(p(x'), y')\wedge unspec_{ATR}(x'))](x,y)$
  \ex $\alpha_{low}'(x)\approx y \defeq \alpha_{low}(x)\approx y](x,y)$
  \end{xlist}
\end{exe}

\noindent The formula in (\ref{blockunspec.def2}a) states that an
element on the vowel tier is associated to an element on the ATR tier in
the output if it is associated to that ATR feature in the input and is
not itself, or if the vocalic element is unspecified and its predecessor
is associated to that same element on the ATR tier in the output. Adding
\enquote{unspec\(_{ATR}(x')\)} to the second disjunct in the lfp
predicate prevents the {[}+ATR{]} feature from spreading to the third
vowel (5), which is already underlyingly associated to {[}-ATR{]}. Thus
in principle, the {[}+low{]} vowel has blocked the spread of {[}+ATR{]},
but only because it is also {[}-ATR{]}. In practice, writing separate
definitions for the associations between the vowel tier and each feature
tier prevents a process on one feature tier from referencing elements on
a different feature tier.

In this section I have shown that describing both unbounded spreading
and blocking transformations over multi-tiered ARs is possible using an
association function (\(\alpha_F(x)\approx y\)). The association
function is similar to the predecessor and successor ordering functions,
p(\(x\)) and s(\(x\)) respectively, because it redefines a binary
relation as a function which takes a term and outputs a term.
Autosegmental association can be defined as a function for ARs of vowel
harmony because it is not possible for a vowel to be associated to both
the + and - minus of a feature so multiple association is only possible
in one direction. Thus \(\alpha_F(x)\approx y\) operates
unidirectionally so \(x\) variables are evaluated as elements on the
vowel tier and \(y\) variables are evaluated as elements on a feature
tier. I defined predicates that allow QFLFP formulas to make reference
to unspecified vowels, which are elements on the vowel tier that are
associated to themselves. The association function is not quite a total
function because it is defined for every element on the vowel tier, but
not for elements on feature tiers.

\section{Conclusion}\label{conclusion}

This paper outlines a method for describing vowel harmony
transformations of both unbounded spreading and blocking patterns over
multi-tiered ARs using quantifier-free least fixed point logic (QFLFP).
It reviews some previous work on both logical descriptions of
phonological transformations and autosegmental theory then extends these
to vowel harmony transformations over multi-tiered autosegmental
representations (ARs). The QFLFP logical descriptions used throughout
this paper reveal the necessity of using functions rather than binary
relations in order to refer to terms, which pick out elements in the
domain. First, the successor ordering relation is translated to a
predecessor function then the use of functions is extended to ARs when
the binary association relation is also translated into an association
function. Unlike with tone, vowel harmony patterns allow autosegmental
association to be defined as a function because multiple association is
only possible in one direction; no vowel can be associated to both the +
and - value of a feature.

Using the association function makes a blocking pattern over
multi-tiered ARs QFLFP definable where the binary association relation
did not. In order to refer to a feature on a different tier---as
descriptions of blocking often do---or a second feature on the same
tier, a FO formula would have to introduce a third variable, which
cannot be done without a quantifier. As a function, though, association
refers to terms which allows for a definition of what have been called
unspecified vowels as elements with a looping association; in other
words elements on the vowel tier that are associated to themselves.
Being able to refer to \enquote{unspecified} vowels effectively
circumvents the problem of not using quantifiers and maintains the
restriction of only having two free variables within the scope of an lfp
predicate.

Chandlee and Jardine
(\protect\hyperlink{ref-chandleejardine2019}{2019}\protect\hyperlink{ref-chandleejardine2019}{b})
claim that unbounded spreading is not AISL because it is not QF
definable, but this paper shows that it is QFLFP definable. It is thus
possible---and other recent work with QFLFP provides further
evidence---that QFLFP describes a superset of the ISL class of functions
established by Chandlee (\protect\hyperlink{ref-chandlee2014}{2014}).
These findings provide valuable insight into the computational
complexity of vowel harmony and phonological transformations in general,
which warrant further research.

Future investigations into the expressivity of QFLFP with respect to
vowel harmony patterns will include developing a description of a
transformation with transparent vowels. In particular, investigating the
proposed underlying forms of vowel harmony transformations with
transparent vowels could provide important insights into the utility of
the representational assumptions held throughout this paper. More
specifically, future work includes and investigation of whether or not
transformations with transparent vowels require that the No Crossing
Constraint (NCC) be violable.

\vspace{2in}

\section{References}\label{references}

\begingroup
\setlength{\parindent}{-0.5in} \setlength{\leftskip}{0.5in}

\hypertarget{refs}{}
\hypertarget{ref-archangelipulleyblank1994}{}
Archangeli, D., \& Pulleyblank, D. (1994). \emph{Grounded phonology}
(Vol. 25). MIT Press.

\hypertarget{ref-chandlee2014}{}
Chandlee, J. (2014). \emph{Strictly local phonological processes}
(PhD thesis). University of Delaware.

\hypertarget{ref-chandleeheinz2018}{}
Chandlee, J., \& Heinz, J. (2018). Strict locality and phonological
maps. \emph{Linguistic Inquiry}, \emph{49}(1), 23--60.

\hypertarget{ref-chandleejardineaisl}{}
Chandlee, J., \& Jardine, A. (2019a). Autosegmental input strictly local
functions. In \emph{Transactions of the association for computational
linguistics} (Vol. 7, pp. 157--168). MIT Press.

\hypertarget{ref-chandleejardine2019}{}
Chandlee, J., \& Jardine, A. (2019b). Quantifier-free monadic
least-fixed point transductions and sequential functions.

\hypertarget{ref-Clements1976}{}
Clements, G. (1976). Vowel harmony in non-linear generative phonology:
An autosegmental model. Bloomington, Indiana University Linguistics
Club.

\hypertarget{ref-Goldsmith1976}{}
Goldsmith, J. (1976). \emph{Autosegmental phonology} (PhD thesis).
Massachusetts Institute of Technology.

\hypertarget{ref-hayes1990}{}
Hayes, B. (1990). Diphthongization and coindexing. \emph{Phonology},
\emph{7}(1), 31--71.

\hypertarget{ref-jardinediss}{}
Jardine, A. (2016). \emph{Locality and non-linear representations in
tonal phonology} (PhD thesis). University of Delaware.

\hypertarget{ref-jardinelocaltone}{}
Jardine, A. (2017). The local nature of tone association patterns.
\emph{Phonology}, \emph{34}(2), 385--405.

\hypertarget{ref-jardineheinz2015}{}
Jardine, A., \& Heinz, J. (2015). A concatenation operation to derive
autosegmental graphs. In \emph{Proceedings of the 14th annual meeting on
the mathematics of language (mol 2015)} (pp. 139--151). Chicago, USA:
Association for Computational Linguistics.

\hypertarget{ref-leben1973}{}
Leben, W. (1973). \emph{Suprasegmental phonology} (PhD thesis).
Massachusetts Institute of Technology.

\hypertarget{ref-libkin2012}{}
Libkin, L. (2012). \emph{Elements of finite model theory}. Springer.

\hypertarget{ref-odden1994}{}
Odden, D. (1994). Adjacency parameters in phonology. \emph{Language},
\emph{70}(2), 289--330.

\hypertarget{ref-ringenvago1998}{}
Ringen, C., \& Vago, R. (1998). Hungarian vowel harmony in optimality.
\emph{Phonology}, \emph{15}, 393--416.

\hypertarget{ref-sagey1986}{}
Sagey, E. (1986). \emph{The representation of features and relations in
non-linear phonology} (PhD thesis). Massachusetts Institute of
Technology.

\endgroup


\end{document}
