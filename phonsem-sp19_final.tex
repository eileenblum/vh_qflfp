\documentclass[,doc,floatsintext]{apa6}
\usepackage{lmodern}
\usepackage{amssymb,amsmath}
\usepackage{ifxetex,ifluatex}
\usepackage{fixltx2e} % provides \textsubscript
\ifnum 0\ifxetex 1\fi\ifluatex 1\fi=0 % if pdftex
  \usepackage[T1]{fontenc}
  \usepackage[utf8]{inputenc}
\else % if luatex or xelatex
  \ifxetex
    \usepackage{mathspec}
  \else
    \usepackage{fontspec}
  \fi
  \defaultfontfeatures{Ligatures=TeX,Scale=MatchLowercase}
\fi
% use upquote if available, for straight quotes in verbatim environments
\IfFileExists{upquote.sty}{\usepackage{upquote}}{}
% use microtype if available
\IfFileExists{microtype.sty}{%
\usepackage{microtype}
\UseMicrotypeSet[protrusion]{basicmath} % disable protrusion for tt fonts
}{}
\usepackage{hyperref}
\hypersetup{unicode=true,
            pdftitle={Vowel harmony with QFLFP},
            pdfauthor={Eileen Blum},
            pdfkeywords={keywords},
            pdfborder={0 0 0},
            breaklinks=true}
\urlstyle{same}  % don't use monospace font for urls
\usepackage{graphicx,grffile}
\makeatletter
\def\maxwidth{\ifdim\Gin@nat@width>\linewidth\linewidth\else\Gin@nat@width\fi}
\def\maxheight{\ifdim\Gin@nat@height>\textheight\textheight\else\Gin@nat@height\fi}
\makeatother
% Scale images if necessary, so that they will not overflow the page
% margins by default, and it is still possible to overwrite the defaults
% using explicit options in \includegraphics[width, height, ...]{}
\setkeys{Gin}{width=\maxwidth,height=\maxheight,keepaspectratio}
\setlength{\emergencystretch}{3em}  % prevent overfull lines
\providecommand{\tightlist}{%
  \setlength{\itemsep}{0pt}\setlength{\parskip}{0pt}}
\setcounter{secnumdepth}{5}
% Redefines (sub)paragraphs to behave more like sections
\ifx\paragraph\undefined\else
\let\oldparagraph\paragraph
\renewcommand{\paragraph}[1]{\oldparagraph{#1}\mbox{}}
\fi
\ifx\subparagraph\undefined\else
\let\oldsubparagraph\subparagraph
\renewcommand{\subparagraph}[1]{\oldsubparagraph{#1}\mbox{}}
\fi

%%% Use protect on footnotes to avoid problems with footnotes in titles
\let\rmarkdownfootnote\footnote%
\def\footnote{\protect\rmarkdownfootnote}

%%% Change title format to be more compact
\usepackage{titling}

% Create subtitle command for use in maketitle
\newcommand{\subtitle}[1]{
  \posttitle{
    \begin{center}\large#1\end{center}
    }
}

\setlength{\droptitle}{-2em}
  \title{Vowel harmony with QFLFP}
  \pretitle{\vspace{\droptitle}\centering\huge}
  \posttitle{\par}
  \author{Eileen Blum\textsuperscript{1}}
  \preauthor{\centering\large\emph}
  \postauthor{\par}
  \date{}
  \predate{}\postdate{}

\shorttitle{VHQFLFP}
\authornote{Eileen Blum is a graduate student in the Department of Linguistics at Rutgers University.


Correspondence concerning this article should be addressed to Eileen Blum, 18 Seminary Place, New Brunswick, NJ 08901. E-mail: eileen.blum@rutgers.edu}
\affiliation{
\vspace{0.5cm}
\textsuperscript{1} Rutgers University}
\keywords{keywords\newline\indent Word count: X}
\usepackage{csquotes}
\usepackage{upgreek}
\captionsetup{font=singlespacing,justification=justified}

\usepackage{longtable}
\usepackage{lscape}
\usepackage{multirow}
\usepackage{tabularx}
\usepackage[flushleft]{threeparttable}
\usepackage{threeparttablex}

\newenvironment{lltable}{\begin{landscape}\begin{center}\begin{ThreePartTable}}{\end{ThreePartTable}\end{center}\end{landscape}}

\makeatletter
\newcommand\LastLTentrywidth{1em}
\newlength\longtablewidth
\setlength{\longtablewidth}{1in}
\newcommand{\getlongtablewidth}{\begingroup \ifcsname LT@\roman{LT@tables}\endcsname \global\longtablewidth=0pt \renewcommand{\LT@entry}[2]{\global\advance\longtablewidth by ##2\relax\gdef\LastLTentrywidth{##2}}\@nameuse{LT@\roman{LT@tables}} \fi \endgroup}


\usepackage{tipa}
\usepackage{gb4e}
\noautomath
\usepackage{tikz}
\usetikzlibrary{matrix}
\tikzset{marked/.style={draw=none, fill=none}}
\usetikzlibrary{topaths}
\usepackage{mathptmx}
\usepackage{moresize}
\setlength{\parindent}{2em}
\def\defeq{\mathrel{\buildrel \mbox{\footnotesize def} \over =}}

\usepackage{amsthm}
\newtheorem{theorem}{Theorem}[section]
\newtheorem{lemma}{Lemma}[section]
\theoremstyle{definition}
\newtheorem{definition}{Definition}[section]
\newtheorem{corollary}{Corollary}[section]
\newtheorem{proposition}{Proposition}[section]
\theoremstyle{definition}
\newtheorem{example}{Example}[section]
\theoremstyle{definition}
\newtheorem{exercise}{Exercise}[section]
\theoremstyle{remark}
\newtheorem*{remark}{Remark}
\newtheorem*{solution}{Solution}
\begin{document}
\maketitle

\section{Introduction}\label{introduction}

\section{Multi-tiered ARs}\label{multi-tiered-ars}

This paper investigates the utility of Quantifier-free Least Fixed Point
logic (QFLFP) for describing vowel harmony transformations over
multi-tiered autosegmental representations (ARs). The multi-tiered ARs
in this paper utilize a version of Hayes
(\protect\hyperlink{ref-hayes1990}{1990})'s bottlebrush theory of vowel
feature representations such that each feature is represented on a
separate tier. Assuming binary vowel features means that each feature
tier includes both the + and - values of a particular feature.
Associations connect the segmental and feature tiers such that each
vowel is directly associated to an element on multiple feature tiers;
thus \enquote{the CV tier\ldots{}resembles the spine of a bottlebrush
with features branching off}(Hayes,
\protect\hyperlink{ref-hayes1990}{1990}). The ARs used throughout this
paper abstract away from consonants and thus the \enquote{spine of the
bottlebrush}(Hayes, \protect\hyperlink{ref-hayes1990}{1990}) only
includes vowels because vowel harmony is assumed to affect only vowel
features.

Much work with ARs includes assumptions about (im)possible autosegmental
structures: the Obligatory Contour Principle (OCP)(Leben,
\protect\hyperlink{ref-leben1973}{1973}) and the No Crossing Constraint
(NCC) (Goldsmith, \protect\hyperlink{ref-Goldsmith1976}{1976}; Sagey,
\protect\hyperlink{ref-sagey1986}{1986}). The investigation into logical
descriptions of vowel harmony transformations carried out here will
provide insights into the necessity of these two constraints on both
input (underlying) and output (surface) structures. Examples of
structures that violate each of these constraints are shown in
(\ref{ncc.ex})-(\ref{ocp.ex}) below.

\begin{exe}
\ex \label{ncc.ex} Violates NCC
  \begin{tikzpicture}[baseline=(current bounding box.north)]
  \matrix [matrix of nodes, row sep=2.5ex, column sep=2.25ex, nodes={text height=1em, text depth=0.5em}] 
  {
* & |(a)| +ATR & |(b)| -ATR \\
  & |(c)| V    & |(d)| V \\
  & |(e)| -low &  \\
  };
  \draw (a.south) -- (d.north);
  \draw (b.south) -- (c.north);
  \draw[thick,black,->] (a) -- (b);
  \draw[thick,black,->] (c) -- (d);
  \draw foreach \x in {c, d} {(\x.south) -- (e.north)};
  \end{tikzpicture}
\end{exe}

\begin{exe}
\ex \label{ocp.ex} Violates OCP
  \begin{tikzpicture}[baseline=(current bounding box.north)]
  \matrix [matrix of nodes, row sep=2.5ex, column sep=2.25ex, nodes={text height=1em, text depth=0.5em}] 
  {
* & |(a)| -ATR & |(b)| -ATR \\
  & |(c)| V    & |(d)| V \\
  & |(e)| -low & |(f)| -low \\
  };
  \draw (a.south) -- (c.north);
  \draw (b.south) -- (d.north);
  \draw (c.south) -- (e.north);
  \draw (d.south) -- (f.north);
  \draw[thick,black,->] (a) -- (b);
  \draw[thick,black,->] (c) -- (d);
  \draw[thick,black,->] (e) -- (f);
  \end{tikzpicture}
\end{exe}

The NCC states that association lines between the segmental tier and a
feature tier never cross. Odden
(\protect\hyperlink{ref-odden1994}{1994}) adds that the NCC can only
evaluate the association between the segmental and one featural tier at
a time. The representation in (\ref{akan.ex3}) violates the NCC because
+ATR precedes -ATR, but is associated to a vowel that is preceded by a
vowel associated to -ATR; this configuration creates visually crossed
association lines.

the OCP stipulates that successive featural elements must be distinct.
The representation in (\ref{akan.ex4}) violates the OCP because on both
the ATR and low feature tiers there are two identical successive
features, -ATR and -low respectively. The OCP in conjunction with FS
results in representations where multiple vowels are associated to a
single feature rather than having multiple successive iterations of the
same feature each associated to a single vowel. An example
representation of an Akan word that satisfies all of the AR properties
discussed here is shown in (\ref{akan.ex}).

Both the NCC and the OCP have also been derived via a concatenation
operation (\(\circ\)) that merges autosegmental \enquote{graph
primitives}(Jardine \& Heinz,
\protect\hyperlink{ref-jardineheinz2015}{2015}, p. 1). An autosegmental
graph primitive consists of an element on the segmental tier, the
elements on each feature tier and the associations between the featural
and segmental tiers. The concatenation operation combines a finite set
of adjacent graph primitives to generate a fully specified AR. For
example, the AR in (\ref{akan.ex}) is derived from the set of graph
primitives in (\ref{concat.ex}). Each primitive in (\ref{concat.ex}) is
concatenated with a single adjacent primitive. If two adjacent
primitives share an identical feature those two features are merged into
one feature with two associations, as in (\ref{akan.ex}). The merging of
identical adjacent features essentially prevents surface ARs from having
multiple iterations of a feature and crossed associations, thus
satisfying both the OCP and the NCC. However, if two segmental elements
are associated to the exact same feature and a different element
intervenes then both iterations of that feature will occur in the
surface AR because only adjacent primitive elements are concatenated and
can thus be merged. This dissertation will show that an intervening
element can be a vowel associated to the same feature with a different
value or a domain boundary. It will further show that a domain boundary
primitive may include that boundary on both segmental and feature tiers.

\begin{exe}
\ex \label{concat.ex} Concatenation of adjacent autosegmental graph primitives \\
  \begin{tikzpicture}[baseline=(current bounding box.north)]
  \matrix [matrix of nodes, row sep=2.5ex, column sep=2.25ex, nodes={text height=1em, text depth=0.5em}] 
  {
  &                                     & |(a)| -ATR &                                     & |(b)| -ATR \\
t & \node {}; \draw(0, 0) circle (3pt); & |(1)| i    & \node {}; \draw(0, 0) circle (3pt); & |(2)| e \\
  &                                     & |(c)| -low &                                     & |(d)| -low \\
  };
  \draw (a.south) -- (1.north);
  \draw (1.south) -- (c.north);
  \draw (b.south) -- (2.north);
  \draw (2.south) -- (d.north);
  \end{tikzpicture}
\end{exe}

This paper will also provide insights into the necessity of the
assumption that surface ARs obey Full Specification (FS)(Clements,
\protect\hyperlink{ref-Clements1976}{1976}), which is held in my current
ongoing work with multi-tiered ARs of vowel harmony. An example of a
structure that violates FS is in (\ref{fs.ex}).

\begin{exe}
\ex \label{akan.ex2} Violates FS
  \begin{tikzpicture}[baseline=(current bounding box.north)]
  \matrix [matrix of nodes, row sep=2.5ex, column sep=2.25ex, nodes={text height=1em, text depth=0.5em}] 
  {
* & |(b)| -ATR & \\
  & |(d)| V    & |(e)| V \\
  & |(f)| -low & \\
  };
  \draw (b.south) -- (d.north);
  \draw (d.south) -- (f.north);
  \draw[thick,black,->] (d) -- (e);
  \draw foreach \x in {d, e} {(\x.south) -- (f.north)};
  \end{tikzpicture}
\end{exe}

FS means that each featural element must be associated to at least one
vowel on the segmental tier and each vowel on the segmental tier must be
associated to at least one element on each featural tier. FS crucially
allows vowels to be associated to multiple featural tiers as is
necessary for each vowel feature to occupy its own tier. The
hypothetical representation in (\ref{akan.ex2}) straighforwardly
violates FS because there is a vowel that is not associated to any
feature on the ATR tier. While both vowels are associated to a single
-low feature, the second vowel is not associated to any feature on the
ATR tier. Since vowel harmony patterns will be analyzed, it will be
assumed that consonants cannot be associated to vowel features and that
FS and vowel harmony in general ignore consonantal elements on the
segmental tier.

A notable effect of FS along with the NCC is that they prevent what have
been called gapped structures (Archangeli \& Pulleyblank,
\protect\hyperlink{ref-archangelipulleyblank1994}{1994}; Ringen \& Vago,
\protect\hyperlink{ref-ringenvago1998}{1998}). A gapped structure is one
in which a feature appears to have skipped over a vowel that it could
potentially be associated to. FS would prevent gapped structures in
which the \enquote{skipped} vowel is not associated to anything on that
particular feature's tier. The NCC would prevent gapped structures in
which the surrounding two vowels are associated to a single feature and
the intervening \enquote{skipped} vowel is associated to a different
feature on the same tier.

\section{QF Logical Transductions}\label{qf-logical-transductions}

Phonological transformations can be described using logical
transductions, which characterize functions. The expressivity of a logic
determines the class of functions that it describes. For example,
previous work has shown that Input Strictly Local (ISL) functions are
describable with quantifier-free (QF) logical transductions (Chandlee,
\protect\hyperlink{ref-chandlee2014}{2014}; Chandlee \& Heinz,
\protect\hyperlink{ref-chandleeheinz2018}{2018}). QF logic is precisely
first-order (FO) logic without quanitfiers (\(\exists\) or \(\forall\)).
In this logic, a term picks out a single element in a model's domain, so
a variable is considered a term and a function applied to a term is also
a term.

\begin{exe}
\ex Terms: \label{terms}
  \begin{xlist}
  \ex a variable $x, y, z, ...$
  \ex a function $f$ applied to a term $t: f(t)$
  \end{xlist}
\end{exe}

QF transductions define an output string in terms of the elements and
relations of the input string using atomic formulas that take terms.

\begin{exe}
\ex QF atomic formulas\label{atom}
  \begin{xlist}
  \ex a unary relation P(t)
  \ex a binary relation R(t$_1$, t$_2$) and
  \ex t$_1 \approx$ t$_2$
  \end{xlist}
\end{exe}

\noindent For an input alphabet \(\Sigma\) and an output alphabet
\(\Gamma\) a logical transduction consists of a copyset C, a set of
unary output condition predicates, and set of unary predicates to label
the output elements. The copyset contains a copy of each input element,
which is given a label via the unary predicates P\(^C_{\gamma}(x)\). The
unary output condition predicates define the conditions under which the
copy of an input element is present in the output. Lastly, QF
transductions are order-preserving as a result of the definition below.

\begin{exe}
\ex\label{order-pres} Let $<$ (and$\leq$) be the transitive (and reflexive) closure of p; build p$'$ such that its transitive closure is $<'$, defined as follows: for all $c, e\in C$
\end{exe}

\hspace{1.85in}
\(\begin{array}{ccc} d^c_1 <' d^e_2 \defeq & x<y & if c\geq e \\  & x\leq y & if c<e \end{array}\)
\vspace{0.2in}

\subsection{Order over strings}\label{order-over-strings}

If we change our binary ordering relation between elements in a string
to a function that applies to an element then we can represent the
successor relation (\(\lhd\)) using a predecessor function (p(\(x\))).
Thus the ordering between elements is characterized by recursively
picking out elements and their predecessors. For example, for the string
in (\ref{string}) the predecessor of 5 is 4 and so p(5) = 4, p(p(5)) =
3, etc. In order to make predecessor a total function, the first element
is defined as that which is its own predecessor.

\begin{exe}
\ex Ordering in a string with the predecessor function \label{string}\\
  \begin{tikzpicture}
\node [circle,draw] (1) at (0,0) {a};
\node [circle,draw] (2) at (1,0) {b};
\node [circle,draw] (3) at (2,0) {b};
\node [circle,draw] (4) at (3,0) {a};
\node [circle,draw] (5) at (4,0) {c};
\draw (1) node [label=below left:1] {};
\draw (2) node [label=below left:2] {};
\draw (3) node [label=below left:3] {};
\draw (4) node [label=below left:4] {};
\draw (5) node [label=below left:5] {};
\path[<-] (1) edge [bend left] node [above] {p} (2);
\path[<-] (2) edge [bend left] node [above] {p} (3);
\path[<-] (3) edge [bend left] node [above] {p} (4);
\path[<-] (4) edge [bend left] node [above] {p} (5);
\path (1) edge [in=190,out=140,loop] node [above] {p} (1);
  \end{tikzpicture}
  
\ex first($x) \defeq$ p($x) \approx x$
\end{exe}

\noindent The string in (\ref{string}) is thus defined over a signature
that consists of the predecessor function and the unary relation that
labels elements with letters of the alphabet: \{p,
P\(_{\sigma\in\Sigma}\)\}, \(\Sigma\) = \{a, b, c\}. The predecessor
function can also be used to define the order between elements on a tier
in autosegmental representations.

\subsection{Association Relation}\label{association-relation}

In addition to ordering between elements on a tier, Chandlee and Jardine
(\protect\hyperlink{ref-chandleejardineaisl}{2019}) introduced the
binary association relation \(\mathcal{A}(x, y)\) between elements on
different tiers. For ARs of tone, \(x\) is evaluated as an element on a
tonal tier and \(y\) as an element on the TBU tier. In order to use this
relation for multi-tiered ARs of vowel harmony, a separate association
relation must be defined between the vowel tier and each feature tier.

\begin{exe}
\ex $\mathcal{A}_F(x, y)$
\ex $\mathcal{A}_G(x, y)$
\end{exe}

\section{Unbounded Spreading}\label{unbounded-spreading}

\subsection{Blocking}\label{blocking}

\section{Association Function}\label{association-function}

\subsection{Blocking}\label{blocking-1}

\subsection{Transparency}\label{transparency}

\subsubsection{The No Crossing
Constraint}\label{the-no-crossing-constraint}

\newpage

\section{References}\label{references}

\begingroup
\setlength{\parindent}{-0.5in} \setlength{\leftskip}{0.5in}

\hypertarget{refs}{}
\hypertarget{ref-archangelipulleyblank1994}{}
Archangeli, D., \& Pulleyblank, D. (1994). \emph{Grounded phonology}
(Vol. 25). MIT Press.

\hypertarget{ref-chandlee2014}{}
Chandlee, J. (2014). \emph{Strictly local phonological processes}
(PhD thesis). University of Delaware.

\hypertarget{ref-chandleeheinz2018}{}
Chandlee, J., \& Heinz, J. (2018). Strict locality and phonological
maps. \emph{Linguistic Inquiry}, \emph{49}(1), 23--60.

\hypertarget{ref-chandleejardineaisl}{}
Chandlee, J., \& Jardine, A. (2019). Autosegmental input strictly local
functions. In \emph{Transactions of the association for computational
linguistics} (Vol. 7, pp. 157--168). MIT Press.

\hypertarget{ref-Clements1976}{}
Clements, G. (1976). Vowel harmony in non-linear generative phonology:
An autosegmental model. Bloomington, Indiana University Linguistics
Club.

\hypertarget{ref-Goldsmith1976}{}
Goldsmith, J. (1976). \emph{Autosegmental phonology} (PhD thesis).
Massachusetts Institute of Technology.

\hypertarget{ref-hayes1990}{}
Hayes, B. (1990). Diphthongization and coindexing. \emph{Phonology},
\emph{7}(1), 31--71.

\hypertarget{ref-jardineheinz2015}{}
Jardine, A., \& Heinz, J. (2015). A concatenation operation to derive
autosegmental graphs. In \emph{Proceedings of the 14th annual meeting on
the mathematics of language (mol 2015)} (pp. 139--151). Chicago, USA:
Association for Computational Linguistics.

\hypertarget{ref-leben1973}{}
Leben, W. (1973). \emph{Suprasegmental phonology} (PhD thesis).
Massachusetts Institute of Technology.

\hypertarget{ref-odden1994}{}
Odden, D. (1994). Adjacency parameters in phonology. \emph{Language},
\emph{70}(2), 289--330.

\hypertarget{ref-ringenvago1998}{}
Ringen, C., \& Vago, R. (1998). Hungarian vowel harmony in optimality.
\emph{Phonology}, \emph{15}, 393--416.

\hypertarget{ref-sagey1986}{}
Sagey, E. (1986). \emph{The representation of features and relations in
non-linear phonology} (PhD thesis). Massachusetts Institute of
Technology.

\endgroup


\end{document}
