\documentclass[,doc,floatsintext]{apa6}
\usepackage{lmodern}
\usepackage{amssymb,amsmath}
\usepackage{ifxetex,ifluatex}
\usepackage{fixltx2e} % provides \textsubscript
\ifnum 0\ifxetex 1\fi\ifluatex 1\fi=0 % if pdftex
  \usepackage[T1]{fontenc}
  \usepackage[utf8]{inputenc}
\else % if luatex or xelatex
  \ifxetex
    \usepackage{mathspec}
  \else
    \usepackage{fontspec}
  \fi
  \defaultfontfeatures{Ligatures=TeX,Scale=MatchLowercase}
\fi
% use upquote if available, for straight quotes in verbatim environments
\IfFileExists{upquote.sty}{\usepackage{upquote}}{}
% use microtype if available
\IfFileExists{microtype.sty}{%
\usepackage{microtype}
\UseMicrotypeSet[protrusion]{basicmath} % disable protrusion for tt fonts
}{}
\usepackage{hyperref}
\hypersetup{unicode=true,
            pdftitle={Vowel Harmony in QFLFP},
            pdfauthor={Eileen Blum},
            pdfkeywords={keywords},
            pdfborder={0 0 0},
            breaklinks=true}
\urlstyle{same}  % don't use monospace font for urls
\usepackage{graphicx,grffile}
\makeatletter
\def\maxwidth{\ifdim\Gin@nat@width>\linewidth\linewidth\else\Gin@nat@width\fi}
\def\maxheight{\ifdim\Gin@nat@height>\textheight\textheight\else\Gin@nat@height\fi}
\makeatother
% Scale images if necessary, so that they will not overflow the page
% margins by default, and it is still possible to overwrite the defaults
% using explicit options in \includegraphics[width, height, ...]{}
\setkeys{Gin}{width=\maxwidth,height=\maxheight,keepaspectratio}
\IfFileExists{parskip.sty}{%
\usepackage{parskip}
}{% else
\setlength{\parindent}{0pt}
\setlength{\parskip}{6pt plus 2pt minus 1pt}
}
\setlength{\emergencystretch}{3em}  % prevent overfull lines
\providecommand{\tightlist}{%
  \setlength{\itemsep}{0pt}\setlength{\parskip}{0pt}}
\setcounter{secnumdepth}{5}
% Redefines (sub)paragraphs to behave more like sections
\ifx\paragraph\undefined\else
\let\oldparagraph\paragraph
\renewcommand{\paragraph}[1]{\oldparagraph{#1}\mbox{}}
\fi
\ifx\subparagraph\undefined\else
\let\oldsubparagraph\subparagraph
\renewcommand{\subparagraph}[1]{\oldsubparagraph{#1}\mbox{}}
\fi

%%% Use protect on footnotes to avoid problems with footnotes in titles
\let\rmarkdownfootnote\footnote%
\def\footnote{\protect\rmarkdownfootnote}

%%% Change title format to be more compact
\usepackage{titling}

% Create subtitle command for use in maketitle
\newcommand{\subtitle}[1]{
  \posttitle{
    \begin{center}\large#1\end{center}
    }
}

\setlength{\droptitle}{-2em}
  \title{Vowel Harmony in QFLFP}
  \pretitle{\vspace{\droptitle}\centering\huge}
  \posttitle{\par}
  \author{Eileen Blum\textsuperscript{1}}
  \preauthor{\centering\large\emph}
  \postauthor{\par}
  \date{}
  \predate{}\postdate{}

\shorttitle{PhonSem Sp2019-VHQFLFP}
\authornote{

Correspondence concerning this article should be addressed to Eileen Blum, 18 Seminary Place. E-mail: eileen.blum@rutgers.edu}
\affiliation{
\vspace{0.5cm}
\textsuperscript{1} Rutgers University}
\abstract{This is the handout for my second presentation in Adam Jardine's Phonology Seminar Spring 2019. 
}
\keywords{keywords\newline\indent Word count: X}
\usepackage{csquotes}
\usepackage{upgreek}
\captionsetup{font=singlespacing,justification=justified}

\usepackage{longtable}
\usepackage{lscape}
\usepackage{multirow}
\usepackage{tabularx}
\usepackage[flushleft]{threeparttable}
\usepackage{threeparttablex}

\newenvironment{lltable}{\begin{landscape}\begin{center}\begin{ThreePartTable}}{\end{ThreePartTable}\end{center}\end{landscape}}

\makeatletter
\newcommand\LastLTentrywidth{1em}
\newlength\longtablewidth
\setlength{\longtablewidth}{1in}
\newcommand{\getlongtablewidth}{\begingroup \ifcsname LT@\roman{LT@tables}\endcsname \global\longtablewidth=0pt \renewcommand{\LT@entry}[2]{\global\advance\longtablewidth by ##2\relax\gdef\LastLTentrywidth{##2}}\@nameuse{LT@\roman{LT@tables}} \fi \endgroup}


\usepackage{tipa}
\usepackage{gb4e}
\noautomath
\usepackage{tikz}
\usetikzlibrary{matrix}
\usetikzlibrary{shapes}
\tikzset{marked/.style={draw=none, fill=none}}
\usepackage{mathptmx}
\usepackage{moresize}
\setlength{\parindent}{2em}
\def\defeq{\mathrel{\buildrel \mbox{\footnotesize def} \over =}}

\usepackage{amsthm}
\newtheorem{theorem}{Theorem}[section]
\newtheorem{lemma}{Lemma}[section]
\theoremstyle{definition}
\newtheorem{definition}{Definition}[section]
\newtheorem{corollary}{Corollary}[section]
\newtheorem{proposition}{Proposition}[section]
\theoremstyle{definition}
\newtheorem{example}{Example}[section]
\theoremstyle{definition}
\newtheorem{exercise}{Exercise}[section]
\theoremstyle{remark}
\newtheorem*{remark}{Remark}
\newtheorem*{solution}{Solution}
\begin{document}
\maketitle

\section{Introduction}\label{introduction}

Last time we talked about vowel harmony transformations with neutral
vowels. We saw how QFLFP can be used to describe spreading and blocking
over strings. Dine, then introduced two-tiered autosegmental
representations of tone and showed that we can also use QFLFP to
describe autosegmental spreading (Chandlee \& Jardine, 2019; Goldsmith,
1976). Today, I will combine these two approaches to show that QFLFP can
be used to describe spreading and blocking of vowel features over
multi-tiered autosegmental representations.

\subsection{Multi-tiered ARs of vowel
harmony}\label{multi-tiered-ars-of-vowel-harmony}

Unlike for tone, we will represent vowel harmony patterns using
autosegmental representations (ARs) with more than two tiers. Ignoring
consonants, we can assume there is one tier, which consists of vowels
and each feature (i.e.~ATR, back, low, round, etc.) is represented on a
separate tier with both its + and - minus values. Assuming a bottlebrush
feature configuraton, each feature tier is connected to the vowel tier
via association, as shown in (\ref{bottlebrush}).

\begin{exe}
\ex \label{bottlebrush} Bottlebrush configuration with a multi-tiered AR \\
  \begin{tikzpicture}[baseline=(current bounding box.north)]
  \matrix [matrix of nodes, row sep=2.5ex, column sep=2.25ex, nodes={text height=1em, text depth=0.5em}] 
  {
               & |(b)| $\pm$back \\
|(a)| $\pm$ATR & \\
|(1)| V        & \\
|(l)| $\pm$low & \\
               & |(r)| $\pm$round \\
  };
  \draw foreach \x in {a, b} {(\x.south) -- (1.north)};
  \draw foreach \x in {l, r} {(1.south) -- (\x.north)};
  \end{tikzpicture}
\end{exe}

\noindent A single vowel is thus necessarily associated to multiple
feature tiers. Assuming both the Obligatory Contour Principle (OCP) and
the No Crossing Constraint (NCC), multi-tiered ARs of vowel harmony can
utilize multiple association, but only in one direction; a single vowel
cannot be associated to both the + and - value of a feature. The
possible surface configurations of association that include both + and -
values on a tier are shown in (\ref{ar.vh}).

\begin{exe}
\ex \label{ar.vh} Possible vowel harmony ARs \\
  \begin{tikzpicture}[baseline=(current bounding box.north)]
  \matrix [matrix of nodes, row sep=2.5ex, column sep=2.25ex, nodes={text height=1em, text depth=0.5em}] 
  {
(a) & |(2)| +F  & |(3)| -F & (b) & |(4)| +F  &           & (c) & |(5)| +F    & |(6)| -F  & (d) & |(7)| +F\\
    & |(b)| V   & |(c)| V  &     & |(d)| V   & |(e)| V   &     & |(f)| V       & |(g)| V   &     & |(h)| V & |(i)| V\\
    & |(g2)| +G &          &     & |(g3)| +G & |(g4)| -G &     & |(g5)| +G    & |(g6)| -G &     & |(g7)| +G\\
  };
   \draw (2.south) -- (b.north);
   \draw (3.south) -- (c.north);
   \draw foreach \x in {b, c} {(\x.south) -- (g2.north)};
   \draw foreach \x in {d, e} {(4.south) -- (\x.north)};
   \draw (d.south) -- (g3.north);
   \draw (e.south) -- (g4.north);
   \draw (5.south) -- (f.north);
   \draw (f.south) -- (g5.north);
   \draw (6.south) -- (g.north);
   \draw (g.south) -- (g6.north);
   \draw foreach \x in {h, i} {(7.south) -- (\x.north)};
   \draw foreach \x in {h, i} {(\x.south) -- (g7.north)};
  \end{tikzpicture}
\end{exe}

\noindent In (\ref{ar.vh}a) two vowels are associated to a single +G
feature and in (\ref{ar.vh}b) two vowels are associated to a single +F
feature. However, if we assume both the OCP and the NCC then on any
feature tier if both the + and - values of a feature are present they
must be associated to different vowels, as in (\ref{ar.vh}a-c). Lastly,
(\ref{ar.vh}d) shows full spreading where both vowels are associated to
a single feature on each tier.

\section{Association Relation}\label{association-relation}

\subsection{Unbounded spreading}\label{unbounded-spreading}

Following Dine's logical characterization of ARs, we could define each
element on each tier and a binary association relation that holds
between them. For example, a vowel harmony pattern like the one in Akan
can be described as an ATR feature spreading from the left until it is
blocked by a +low vowel (Clements, 1976). A word with no +low vowels
will show unbounded feature spreading, as in (\ref{akan.spread}).

\begin{exe}
\ex \label{akan.spread} Unbounded Feature Spreading \\
  \begin{tikzpicture}[baseline=(current bounding box.north)]
  \matrix [matrix of nodes, row sep=2.5ex, column sep=2.25ex, nodes={text height=1em, text depth=0.5em}] 
  {
|(a)| +ATR &         &         &           & |(c)| +ATR\\
|(1)| V    & |(2)| V & |(3)| V & $\mapsto$ & |(4)| V    & |(5)| V & |(6)| V\\
|(b)| -low &             &     &           & |(d)| -low\\
  };
  \draw (a.south) -- (1.north);
  \draw (1.south) -- (b.north);
  \draw foreach \x in {4, 5, 6} {(c.south) -- (\x.north)};
  \draw foreach \x in {4, 5, 6} {(\x.south) -- (d.north)};
  \end{tikzpicture}
\end{exe}

\noindent Unbounded feature spreading is straightforwardly captured with
QFLFP because all surface features are found underlyingly and the
associations can be said to spread iteratively. We thus define a unary
relation for vowels and each feature (separate ones for + and - values).

\begin{exe}
\ex\label{qflfp.spread} $\langle$D; p, s, $\mathcal{A}$, P$_V$, P$_{+ATR}$, P$_{-ATR}$, P$_{+low}$, P$_{-low}\rangle$ 
\end{exe}

\noindent P\('_{V}(x) \defeq\) P\(_{V}(x)\)\newline
P\('_{+ATR}(x) \defeq\) P\(_{+ATR}(x)\) \hspace{0.5in}
P\('_{-ATR}(x) \defeq\) P\(_{-ATR}(x)\)\newline
P\('_{+low}(x) \defeq\) P\(_{+low}(x)\) \hspace{0.5in}
P\('_{-low}(x) \defeq\) P\(_{-low}(x)\)\newline

\noindent We can also define two association relations, between the
vowel tier and each feature tier.

\noindent \(\mathcal{A}_{ATR}'(x, y) \defeq [lfp\mathcal{A}(x', y') \vee R(x', p(y'))](x, y)\)\newline
\(\mathcal{A}_{low}'(x, y) \defeq [lfp\mathcal{A}(x', y') \vee R(x', p(y'))](x, y)\)

\subsection{Blocking}\label{blocking}

However, Akan ATR harmony is blocked by a +low vowel so we need to add a
restriction to these association definitions. If we stipulate that +low
vowels are underlyingly specified for ATR, we will see something like
(\ref{akan.block}).

\begin{exe}
  \ex \label{akan.block} Blocking\\
  \begin{tikzpicture}[baseline=(current bounding box.north)]
  \matrix [matrix of nodes, row sep=2.5ex, column sep=2.25ex, nodes={text height=1em, text depth=0.5em}] 
  {
|(a1)| +ATR &         & |(b1)| -ATR &         &           & |(c1)| +ATR &         & |(d1)| -ATR\\
|(1)| V  & |(2)| V & |(3)| V      & |(4)| V & $\mapsto$ & |(5)| V     & |(6)| V & |(7)| V & |(8)| V\\
|(a2)| -low &         & |(b2)| +low &         &           & |(c2)| -low &         & |(d2)| +low\\
  };
  \draw (a1.south) -- (1.north);
  \draw (1.south) -- (a2.north);
  \draw (b1.south) -- (3.north);
  \draw (3.south) -- (b2.north);
  \draw foreach \x in {5, 6} {(c1.south) -- (\x.north)};
  \draw foreach \x in {5, 6} {(\x.south) -- (c2.north)};
  \draw foreach \x in {7, 8} {(d1.south) -- (\x.north)};
  \draw foreach \x in {7, 8} {(\x.south) -- (d2.north)};
  \end{tikzpicture}
\end{exe}

\noindent Again, each feature and all the vowels only appear in the
output structure if they are present in the input structure. The idea
that we want to capture is that a feature spreads from one vowel onto
successive \emph{unassociated} vowels and stops when it reaches another
vowel that is \emph{underlyingly associated}. Is it possible to
incorporate blocking into the definitions of the binary association
relation?

\begin{exe}
\ex\label{qflfp.spread} $\langle$D; p, s, $\mathcal{A}$, $P_V$, $P_{+ATR}$, $P_{-ATR}$, $P_{+low}$, $P_{-low}\rangle$ 
\end{exe}

\noindent \(P'_{V}(x) \defeq P_{V}(x)\)\newline
\(P'_{+ATR}(x) \defeq P_{+ATR}(x)\) \hspace{0.5in}
\(P'_{-ATR}(x) \defeq P_{-ATR}(x)\)\newline
\(P'_{+low}(x) \defeq P_{+low}(x)\) \hspace{0.5in}
\(P'_{-low}(x) \defeq P_{-low}(x)\)\newline
\(\mathcal{A}_{ATR}'(x, y) \defeq [lfp\mathcal{A}(x', y')\vee (R(x', p(y')) \wedge\)
\vspace{1in}

\noindent \(\mathcal{A}_{low}'(x, y) \defeq [lfp\mathcal{A}(x', y')\vee (R(x', p(y')) \wedge\)

\newpage

\section{Association as a function}\label{association-as-a-function}

\subsection{Unbounded spreading}\label{unbounded-spreading-1}

An alternative approach would be to describe the associations between
the vowel tier and each feature tier as a function. Because multiple
association is possible with multi-tiered ARs, association can only be a
function if it is unidirectional. If you look back at the possible
structures in (\ref{ar.vh}), in which direction could a function that
holds between tiers operate?

A separate association function (let's call it \(\alpha\)) must be
defined for each feature tier, but since they are all the same type of
function (\(\alpha(x)\approx y\)) they will work in the same way. An
example transformation for Akan is given in (\ref{akan.func}) using the
unidirectional association function represented with arrows from vowels
to a feature. In order to evaluate \(\alpha(x)\approx y\), the variable
\(x\) would be used for an element on the vowel tier and \(y\) for an
element on a feature tier.

\begin{exe}
\ex \label{akan.func} Association as a unidirectional function \\
  \begin{tikzpicture}[baseline=(current bounding box.north)]
  \matrix [matrix of nodes, row sep=2.5ex, column sep=2.25ex, nodes={text height=1em, text depth=0.5em}] 
  {
|(a)| +ATR$_1$ &             &             &           & |(c)| +ATR$_6$\\
|(1)| V$_2$    & |(2)| V$_3$ & |(3)| V$_4$ & $\mapsto$ & |(4)| V$_7$ & |(5)| V$_8$ & |(6)| V$_9$ \\
|(b)| -low$_5$ &             &             &           & |(d)| -low$_{10}$\\
  };
  \draw[black,<-] (a.south) -- (1.north);
  \draw[black,->] (1.south) -- (b.north);
  \draw[black,<-] foreach \x in {4, 5, 6} {(c.south) -- (\x.north)};
  \draw[black,->] foreach \x in {4, 5, 6} {(\x.south) -- (d.north)};
  \end{tikzpicture}
\end{exe}

\noindent We can use the same definitions for unary P relations as
before. Can we also define the output associations for unbounded
spreading in (\ref{akan.func}) in terms of the input associations?

\begin{exe}
\ex\label{function.spread} $\langle$D; p, s, $\alpha$, $P_V$, $P_{+ATR}$, $P_{-ATR}$, $P_{+low}$, $P_{-low}\rangle$ 
\end{exe}

\noindent \(P'_{V}(x) \defeq P_{V}(x)\)\newline
\(P'_{+ATR}(x) \defeq P_{+ATR}(x)\) \hspace{0.5in}
\(P'_{-ATR}(x) \defeq P_{-ATR}(x)\)\newline
\(P'_{+low}(x) \defeq P_{+low}(x)\) \hspace{0.5in}
\(P'_{-low}(x) \defeq P_{-low}(x)\)\newline
\(\alpha_{ATR}'(x)\approx y \defeq\) \vspace{1in}

\noindent \(\alpha_{low}'(x)\approx y \defeq\)

\newpage

\noindent Let's evaluate these functions over the input structure
according to our definitions and see if the results match the output
structure in (\ref{akan.func}). Recall that in order for \(\alpha\) to
be a function there must be only one \(y\) for each \(x\). Multiple
\(x\)s can map to the same \(y\), but not vice versa.

\noindent \(\alpha_{ATR}(2)\approx\) \hspace{0.75in}
\(\alpha_{ATR}(3)\approx\) \hspace{0.75in} \(\alpha_{ATR}(4)\approx\)
\vspace{0.5in}

\noindent \(\alpha_{low}(2)\approx\) \hspace{0.75in}
\(\alpha_{low}(3)\approx\) \hspace{0.75in} \(\alpha_{low}(4)\approx\)
\vspace{0.5in}

\subsection{Total Function}\label{total-function}

Remember the string relations that we are now considering as functions:
predecessor and successor. In the current system, these are both total
functions over strings because each element in the input string has a
predecessor and a successor, including the first and last one
respectively. In other words, the first element in a string is its own
predecessor and the last element in a string is its own successor. Could
we extend this idea to ARs? For ARs, how could we make association a
total function? Let's look at (\ref{akan.func}) again repeated here in
(\ref{ex.func}).

\begin{exe}
\ex \label{ex.func} Association as a unidirectional function \\
  \begin{tikzpicture}[baseline=(current bounding box.north)]
  \matrix [matrix of nodes, row sep=2.5ex, column sep=2.25ex, nodes={text height=1em, text depth=0.5em}] 
  {
|(a)| +ATR$_1$ &             &             &           & |(c)| +ATR$_6$\\
|(1)| V$_2$    & |(2)| V$_3$ & |(3)| V$_4$ & $\mapsto$ & |(4)| V$_7$ & |(5)| V$_8$ & |(6)| V$_9$ \\
|(b)| -low$_5$ &             &             &           & |(d)| -low$_{10}$\\
  };
  \draw[black,<-] (a.south) -- (1.north);
  \draw[black,->] (1.south) -- (b.north);
  \draw[black,<-] foreach \x in {4, 5, 6} {(c.south) -- (\x.north)};
  \draw[black,->] foreach \x in {4, 5, 6} {(\x.south) -- (d.north)};
  \end{tikzpicture}
\end{exe}

\noindent First, all elements on the vowel tier must be associated to
something. Notice the input AR is underspecified because the vowels 2
and 3 are not associated to any features. We can solve this in the same
way that we did for strings. We can say that unspecified vowels in the
input are associated to themselves and then our definitions for output
association will overwrite the input associations to spread the feature
to all vowels. This will make the unbounded spreading ARs look like
(\ref{total.func})

\newpage

\begin{exe}
\ex \label{total.func} Association as a total function \\
  \begin{tikzpicture}[baseline=(current bounding box.north)]
  \matrix [matrix of nodes, row sep=2.5ex, column sep=2.25ex, nodes={text height=1em, text depth=0.5em}] 
  {
|(a)| +ATR$_1$ &             &             &           & |(c)| +ATR$_6$\\
|(1)| V$_2$    & |(2)| V$_3$ & |(3)| V$_4$ & $\mapsto$ & |(4)| V$_7$ & |(5)| V$_8$ & |(6)| V$_9$ \\
|(b)| -low$_5$ &             &             &           & |(d)| -low$_{10}$\\
  };
  \draw[black,<-] (a.south) -- (1.north);
  \draw[black,->] (1.south) -- (b.north);
  \draw[black,<-] foreach \x in {4, 5, 6} {(c.south) -- (\x.north)};
  \draw[black,->] foreach \x in {4, 5, 6} {(\x.south) -- (d.north)};
  \path (2) edge [loop above] (2);
  \path (3) edge [loop above] (3);
  \path (2) edge [loop below] (2);
  \path (3) edge [loop below] (3);
  \end{tikzpicture}
\end{exe}

\noindent Including looping associations in the input will now allow us
to refer to unspecified vowels, which we could not do with the binary
association relation.

\subsection{Blocking}\label{blocking-1}

Lastly, we cannot forget that Akan has blocking. The major advantage of
defining association as a function rather than a binary relation is that
we can use QFLFP to describe feature spreading with blocking. We can now
write the complete AR transformation for blocking by adding to
(\ref{akan.block}), as in (\ref{totalfunc.block}).

\begin{exe}
  \ex \label{totalfunc.block} Blocking\\
  \begin{tikzpicture}[baseline=(current bounding box.north)]
  \matrix [matrix of nodes, row sep=2.5ex, column sep=2.25ex, nodes={text height=1em, text depth=0.5em}] 
  {
|(a1)| +ATR &         & |(b1)| -ATR &         &           & |(c1)| +ATR &         & |(d1)| -ATR\\
|(1)| V     & |(2)| V & |(3)| V     & |(4)| V & $\mapsto$ & |(5)| V     & |(6)| V & |(7)| V & |(8)| V\\
|(a2)| -low &         & |(b2)| +low &         &           & |(c2)| -low &         & |(d2)| +low\\
  };
  \draw (a1.south) -- (1.north);
  \draw (1.south) -- (a2.north);
  \draw (b1.south) -- (3.north);
  \draw (3.south) -- (b2.north);
  \draw foreach \x in {5, 6} {(c1.south) -- (\x.north)};
  \draw foreach \x in {5, 6} {(\x.south) -- (c2.north)};
  \draw foreach \x in {7, 8} {(d1.south) -- (\x.north)};
  \draw foreach \x in {7, 8} {(\x.south) -- (d2.north)};
  \path (2) edge [loop above] (2);
  \path (4) edge [loop above] (4);
  \path (2) edge [loop below] (2);
  \path (4) edge [loop below] (4);
  \end{tikzpicture}
\end{exe}

\noindent The question remains whether or not we can define the
association function using QFLFP to incorporate blocking. Using QFLFP we
can now describe a transformation that spreads a feature association
from one vowel only to an adjacent unspecified vowel. First, we can
define an unspecified element in the same way that we defined the first
and last elements in a string. How would you define the unspec(\(x\))
predicate?

unspec\(_{ATR}(x) \defeq\) \vspace{0.5in}

unspec\(_{low}(x) \defeq\) \vspace{0.5in}

\noindent Next, we can define the unary relations for each element in
the same way we have been so far.

\begin{exe}
\ex\label{function.block} $\langle$D; p, s, $\alpha$, $P_V$, $P_{+ATR}$, $P_{-ATR}$, $P_{+low}$, $P_{-low}\rangle$ 
\end{exe}

\noindent \(P'_{V}(x) \defeq P_{V}(x)\)\newline
\(P'_{+ATR}(x) \defeq P_{+ATR}(x)\) \hspace{0.5in}
\(P'_{-ATR}(x) \defeq P_{-ATR}(x)\)\newline
\(P'_{+low}(x) \defeq P_{+low}(x)\) \hspace{0.5in}
\(P'_{-low}(x) \defeq P_{-low}(x)\)\newline

\noindent Lastly using the predicate unspec(\(x\)) defined above, how
can we define the two association functions needed to describe the Akan
vowel harmony pattern with blocking?

\noindent \(\alpha_{ATR}'(x)\approx y \defeq\) \vspace{1in}

\noindent \(\alpha_{low}'(x)\approx y \defeq\) \vspace{2in}

\section{References}\label{references}

\begingroup
\setlength{\parindent}{-0.5in} \setlength{\leftskip}{0.5in}

\hypertarget{refs}{}
\hypertarget{ref-chandleejardineaisl}{}
Chandlee, J., \& Jardine, A. (2019). Autosegmental input strictly local
functions.

\hypertarget{ref-Clements1976}{}
Clements, G. (1976). Vowel harmony in non-linear generative phonology:
An autosegmental model. Bloomington, Indiana University Linguistics
Club.

\hypertarget{ref-Goldsmith1976}{}
Goldsmith, J. (1976). \emph{Autosegmental phonology} (PhD thesis).
Massachusetts Institute of Technology.

\endgroup


\end{document}
