\documentclass[ignorenonframetext,]{beamer}
\setbeamertemplate{caption}[numbered]
\setbeamertemplate{caption label separator}{: }
\setbeamercolor{caption name}{fg=normal text.fg}
\beamertemplatenavigationsymbolsempty
\usepackage{lmodern}
\usepackage{amssymb,amsmath}
\usepackage{ifxetex,ifluatex}
\usepackage{fixltx2e} % provides \textsubscript
\ifnum 0\ifxetex 1\fi\ifluatex 1\fi=0 % if pdftex
  \usepackage[T1]{fontenc}
  \usepackage[utf8]{inputenc}
\else % if luatex or xelatex
  \ifxetex
    \usepackage{mathspec}
  \else
    \usepackage{fontspec}
  \fi
  \defaultfontfeatures{Ligatures=TeX,Scale=MatchLowercase}
\fi
\usetheme[]{Boadilla}
\usecolortheme{beaver}
% use upquote if available, for straight quotes in verbatim environments
\IfFileExists{upquote.sty}{\usepackage{upquote}}{}
% use microtype if available
\IfFileExists{microtype.sty}{%
\usepackage{microtype}
\UseMicrotypeSet[protrusion]{basicmath} % disable protrusion for tt fonts
}{}
\newif\ifbibliography
\hypersetup{
            pdftitle={Phonology Seminar-final\_sp2019},
            pdfauthor={Eileen Blum},
            pdfborder={0 0 0},
            breaklinks=true}
\urlstyle{same}  % don't use monospace font for urls

% Prevent slide breaks in the middle of a paragraph:
\widowpenalties 1 10000
\raggedbottom

\AtBeginPart{
  \let\insertpartnumber\relax
  \let\partname\relax
  \frame{\partpage}
}
\AtBeginSection{
  \ifbibliography
  \else
    \let\insertsectionnumber\relax
    \let\sectionname\relax
    \frame{\sectionpage}
  \fi
}
\AtBeginSubsection{
  \let\insertsubsectionnumber\relax
  \let\subsectionname\relax
  \frame{\subsectionpage}
}

\setlength{\parindent}{0pt}
\setlength{\parskip}{6pt plus 2pt minus 1pt}
\setlength{\emergencystretch}{3em}  % prevent overfull lines
\providecommand{\tightlist}{%
  \setlength{\itemsep}{0pt}\setlength{\parskip}{0pt}}
\setcounter{secnumdepth}{0}
\usepackage{tipa}
\usepackage{gb4e}
\noautomath
\usepackage{tikz}
\usetikzlibrary{matrix}
\tikzset{marked/.style={draw=none, fill=none}}
\usepackage{mathptmx}
\usepackage{moresize}
\setlength{\parindent}{2em}
\usepackage{multicol}
\usecolortheme[RGB={204,0,51}]{structure}

\title{Phonology Seminar-final\_sp2019}
\author{Eileen Blum}
\date{4/26/2019}

\begin{document}
\frame{\titlepage}

\begin{frame}{Introduction}

\begin{itemize}
\tightlist
\item
  Vowel harmony patterns with neutral vowels utilize:

  \begin{itemize}
  \tightlist
  \item
    unbounded spreading
  \item
    blocking
  \item
    transparent vowels
  \end{itemize}
\item
  Unbounded spreading and blocking are QFLFP definable over multi-tiered
  autosegmental representations
\end{itemize}

\end{frame}

\begin{frame}{Multi-tiered ARs}

\begin{itemize}
\tightlist
\item
  Bottle brush representation
\item
  Obey OCP, NCC
\end{itemize}

Bottlebrush configuration with a multi-tiered AR

\begin{tikzpicture}[baseline=(current bounding box.north)]
  \matrix [matrix of nodes, ampersand replacement=\&, row sep=1.25ex, column sep=2.25ex, nodes={text height=0.75em, text depth=0.25em}] 
  {
               \& |(b)| $\pm$back \\
|(a)| $\pm$ATR \& \\
|(1)| V        \& \\
|(l)| $\pm$low \& \\
               \& |(r)| $\pm$round \\
  };
  \draw (a.south) -- (1.north);
  \draw (b.south) -- (1.north);
  \draw (1.south) -- (l.north);
  \draw (1.south) -- (r.north);
\end{tikzpicture}

\end{frame}

\begin{frame}{Association Function}

\begin{itemize}
\tightlist
\item
  Vowels cannot be associated to multiple feature values on the same
  tier

  \begin{itemize}
  \tightlist
  \item
    unidrectional association function from vowel(s) to a feature
  \end{itemize}
\end{itemize}

Possible vowel harmony ARs

\begin{tikzpicture}[baseline=(current bounding box.north)]
  \matrix [matrix of nodes, ampersand replacement=\&, row sep=2.5ex, column sep=2.25ex, nodes={text height=1em, text depth=0.5em}] 
  {
(a) \& |(2)| +F  \& |(3)| -F \& (b) \& |(4)| +F  \&           \& (c) \& |(5)| +F    \& |(6)| -F  \& (d) \& |(7)| +F\\
    \& |(b)| V   \& |(c)| V  \&     \& |(d)| V   \& |(e)| V   \&     \& |(f)| V       \& |(g)| V   \&     & |(h)| V \& |(i)| V\\
    \& |(g2)| +G \&          \&     \& |(g3)| +G \& |(g4)| -G \&     \& |(g5)| +G    \& |(g6)| -G \&     \& |(g7)| +G\\
  };
  \draw (2.south) -- (b.north);
  \draw (3.south) -- (c.north);
  \draw foreach \x in {b, c} {(\x.south) -- (g2.north)};
  \draw foreach \x in {d, e} {(4.south) -- (\x.north)};
  \draw (d.south) -- (g3.north);
  \draw (e.south) -- (g4.north);
  \draw (5.south) -- (f.north);
  \draw (f.south) -- (g5.north);
  \draw (6.south) -- (g.north);
  \draw (g.south) -- (g6.north);
  \draw foreach \x in {h, i} {(7.south) -- (\x.north)};
  \draw foreach \x in {h, i} {(\x.south) -- (g7.north)};
\end{tikzpicture}

\end{frame}

\end{document}
